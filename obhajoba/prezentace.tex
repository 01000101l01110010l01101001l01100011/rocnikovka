\documentclass[aspectratio=169]{beamer}
\usetheme{CambridgeUS}
\usecolortheme{spruce}
\setbeamersize{text margin left=1cm,text margin right=1cm}
\usepackage[utf8]{inputenc}
\usepackage[T1]{fontenc}
\usepackage[czech]{babel}
\usepackage{csquotes}
\usepackage{amsmath}
\usepackage{tikz}
\usefonttheme{professionalfonts}
\usetikzlibrary{%
 positioning, graphs, graphs.standard, shapes.geometric,quotes, calc, perspective, shapes, arrows.meta
}
\usepackage{pgfplots}
\usepackage{tikz-3dplot}
\usepackage{caption}
\usepackage{subcaption}

\definecolor{myred}{RGB}{170, 55, 55}
\definecolor{myblue}{RGB}{88,118,199}
\definecolor{mygreen}{RGB}{55, 120, 55}
\definecolor{graph}{RGB}{102,138,224}
\definecolor{graph2}{RGB}{224,102,138}
\definecolor{myorange}{RGB}{237, 120, 36}


\definecolor{green1}{RGB}{5,147,84}
\definecolor{green2}{RGB}{2,123,54}
\definecolor{green3}{RGB}{29,105,68}

\definecolor{white1}{RGB}{220,250,250}
\definecolor{vyzkum_color}{RGB}{119,96,210}

\setbeamercolor{structure}{fg=mygreen} % Change the color of structural elements to "myred"
% \setbeamercolor{block title}{bg=blue!70!black, fg=white} % Change the color of block titles
\setbeamercolor{block title}{bg=myblue, fg=white}

\setbeamercolor{palette primary}{bg=green1}
\setbeamercolor{palette secondary}{bg=green2}
\setbeamercolor{palette tertiary}{bg=green3}


\setbeamercolor{title in head/foot}{fg=white1}
\setbeamercolor{author in head/foot}{fg=white1}
\setbeamercolor{date in head/foot}{fg=white1}
\setbeamercolor{title}{fg=white}
\setbeamercolor{section in head/foot}{fg=white1}
\setbeamercolor{subsection in head/foot}{fg=white1}

\newenvironment<>{vyzkum}[1]{%
  \setlength{\textwidth}{1\textwidth}
  \setbeamercolor{block title}{fg=white,bg=vyzkum_color}%
  \begin{block}#2{#1}}{\end{block}}
\newenvironment<>{random}[1]{%
    \setlength{\textwidth}{0.75\textwidth}
    \setbeamercolor{block title}{fg=white,bg=green1}%
    \begin{block}#2{#1}}{\end{block}}

\newenvironment<>{informace}[1]{%
    \setlength{\textwidth}{1\textwidth}
    \setbeamercolor{block title}{fg=white,bg=myblue}%
    \begin{block}#2{#1}}{\end{block}}

\newenvironment<>{otazka}[1]{%
    \setlength{\textwidth}{1\textwidth}
    \setbeamercolor{block title}{fg=white,bg=myorange}%
    \begin{block}#2{#1}}{\end{block}}

\usepackage{enumitem,amssymb}
\newlist{todolist}{itemize}{2}
\setlist[todolist]{label=$\square$}
\usepackage{pifont}
\newcommand{\cmark}{\ding{51}}%
\newcommand{\xmark}{\ding{55}}%
\newcommand{\done}{\rlap{$\square$}{\raisebox{2pt}{\large\hspace{1pt}\cmark}}%
\hspace{-2.5pt}}
\newcommand{\wontfix}{\rlap{$\square$}{\large\hspace{1pt}\xmark}}

\newtheorem{definice}{Definice}
\newcommand{\R}{\mathbb{R}}
\beamertemplatenavigationsymbolsempty


% \usepackage[
%  backend=biber,
%  style=alphabetic,
%  sorting=ynt
% ]{biblatex}
% \addbibresource{refs.bib}


\begin{document}

\title[Ročníková práce]{Hledání polytopu maximální dimenze a minimálního obvodu s~vrcholy v dané množině bodů}
\author{Eric Dusart}
\date{17. května 2024}

\begin{frame}
  \titlepage
\end{frame}

\begin{frame}
  \frametitle{Obsah}
  \tableofcontents
\end{frame}

\section{Polytop}
\begin{frame}
  \frametitle{Co je to polytop?}
  \begin{informace}{Informace o polytopu}
    \begin{itemize}[label=\textbullet]
      \item \emph{Polytop} dimenze $n \in \mathbb{N}$ je uzavřená podmnožina $P \subseteq \R^{n}$.
      \item Polytop maximální dimenze a minimálního obvodu má $n+1$ vrcholů.
      \item Neexistuje nadrovina (podprostor dimenze $n - 1$), která by obsahovala všechny vrcholy polytopu.
    \end{itemize}
  \end{informace}

  \begin{figure}
    \centering
    \begin{subfigure}[b]{0.3\textwidth}
      \centering
      \begin{tikzpicture}
        \draw[myred, thick] (0,0) -- (2,0);
        \fill[myred] (0,0) circle (1pt);
        \fill[myred] (2,0) circle (1pt);
      \end{tikzpicture}
      \caption{Úsečka}
      \label{fig:line_segment}
    \end{subfigure}
    \begin{subfigure}[b]{0.3\textwidth}
      \centering
      \begin{tikzpicture}
        \draw[myblue, thick] (0,0) -- (1,1) -- (2,0) -- cycle;
        \fill[myblue] (0,0) circle (1pt);
        \fill[myblue] (1,1) circle (1pt);
        \fill[myblue] (2,0) circle (1pt);
      \end{tikzpicture}
      \caption{Mnohoúhelník}
      \label{fig:polygon}
    \end{subfigure}
    \begin{subfigure}[b]{0.3\textwidth}
      \centering
      \begin{tikzpicture}[3d view]
        \draw[green2, thick, dotted] (1,0,0) -- (0,1,0);
        \draw[green2, thick] (0,1,0) -- (0,0,2);
        \draw[green2, thick] (0,-1,0) -- (1,0,0);
        \draw[green2, thick] (1,0,0) -- (0,0,2);
        \draw[green2, thick] (0,-1,0) -- (0,0,2);
        \draw[green2, thick] (0,1,0) -- (0,-1,0);
        \fill[green3] (1,0,0) circle (1pt);
        \fill[green3] (0,1,0) circle (1pt);
        \fill[green3] (0,-1,0) circle (1pt);
        \fill[green3] (1,0,0) circle (1pt);
        \fill[green3] (0,0,1.99) circle (1pt);
      \end{tikzpicture}
      \caption{Mnohostěn}
      \label{fig:polygonds}
    \end{subfigure}
  \end{figure}
\end{frame}

\section{Moje práce}
\begin{frame}
  \frametitle{Moje práce}

  \begin{vyzkum}{Výzkumná otázka}
    Jak najít polytop maximální dimenze a minimálního obvodu s vrcholy v dané množině bodů?
  \end{vyzkum}

  \vspace{1ex}
  \begin{minipage}{0.5\textwidth}
      \textbf{Rozdělení práce:}
      \begin{todolist}
        \item[\done] Problém v 1D, 2D a $n$D
        \begin{todolist}
          \item[\done] Najít algoritmus.
          \item[\done] Dokázat, že funguje.
          \item[\done] Naprogramovat algoritmus.
        \end{todolist}
      \end{todolist}
  \end{minipage}
  \begin{minipage}{0.5\textwidth}
      \begin{random}{Proč jsem si vybral toto téma:}
        \begin{itemize}[label=\textbullet]
          \item Zájem o matematiku.
          \item Trojúhelníky ve 2D.
          \item Problém v $n$ dimenzích.
        \end{itemize}
      \end{random}
  \end{minipage}
\end{frame}

\section{Otázky}
% \begin{frame}
%   \frametitle{Otázky}
%   \begin{itemize}[label=\textbullet]
%     \item Proč je užití Dijkstrova algoritmu v dvoudimenzionální variantě problému stejně efektivní jako procházení všech
%     možností?
%     \item Co má větší vliv na časovou náročnost algoritmu? Dimenze či počet bodů? V jakém smyslu a proč?
%     \item Co když leží všechny vstupní body v jedné nadrovině?
%     \item Můžeš odhadnou náročnost tvého algoritmu pro $n=10$ (alespoň přibližně)?
%   \end{itemize}
% \end{frame}
% \subsection{Otázka 1}
\begin{frame}
  \begin{otazka}{Proč je užití Dijkstrova algoritmu v dvoudimenzionální variantě problému stejně efektivní jako procházení všech možností?}
    Pokud máme zvolený bod $a$ a hledáme nejkratší cestu do bodu $b$, tak Dijkstrův algoritmus bude fungovat následovně:
    \begin{itemize}[label=\textbullet]
      \item Vybere bod $a$ jako počáteční a přiřadí všem ostatním bodům vzdálenost $\infty$.
      \item $\forall u \in V \setminus \{a\}$ zkontroluje, jestli $w(a, u) < \infty$, a pokud ano, vzdálenost změní.
      \item \dots
    \end{itemize}

    \begin{figure}[H]
      \centering
      \begin{tikzpicture}[scale=1.5]
        % Vertices
        \node[draw,circle,fill=mygreen,inner sep=0pt,minimum size=4pt] (v1) at (1, 0) {}; %START
        \coordinate[label=left:$a$] (A) at (1,0);
        \node[draw,circle,fill=purple,inner sep=0pt,minimum size=4pt] (v2) at (2, 0.5) {};
        \node[draw,circle,fill=myblue,inner sep=0pt,minimum size=4pt] (v3) at (4, 0) {};
        \node[draw,circle,fill=myred,inner sep=0pt,minimum size=4pt] (v4) at (2.5, 1.5) {}; %END
        \coordinate[label=left:$b$] (A) at (2.5,1.5);
        \node[draw,circle,fill=pink,inner sep=0pt,minimum size=4pt] (v5) at (3, 0.5) {};
        % Edges
        \draw[line width=0.5pt, dashed] (v1) -- (v4);
        \draw[line width=0.5pt] (v3) -- (v5);
        \draw[line width=0.5pt] (v2) -- (v3);
        \draw[line width=0.5pt] (v2) -- (v5);
        \draw[purple, line width=0.75pt] (v1) -- (v2);
        \draw[myblue, line width=0.75pt] (v1) -- (v3);
        \draw[pink, line width=0.75pt] (v1) -- (v5);
        \draw[purple, line width=0.75pt] (v2) -- (v4);
        \draw[myblue, line width=0.75pt] (v3) -- (v4);
        \draw[pink, line width=0.75pt] (v4) -- (v5);

      \end{tikzpicture}
      \caption{Graf $K_5$}
    \end{figure}

  \end{otazka}
\end{frame}
% \subsection{Otázka 2}
\begin{frame}
  \begin{otazka}{Co když leží všechny vstupní body v jedné nadrovině?}
    \begin{itemize}[label=\textbullet]
      \item Hledaný polytop neexistuje.
    \end{itemize}
    \tdplotsetmaincoords{60}{22.5} % Set the viewing angle
    \begin{figure}
      \begin{subfigure}{0.4\textwidth}
        \begin{tikzpicture}[scale=2,tdplot_main_coords]
          % Define coordinates
          \coordinate (a) at (0.5,0.5,0);
          \coordinate (b) at (1.5,0.5,0);
          \coordinate (c) at (1.75,1.75,0);
          \coordinate (d) at (0.5,1.5,0);

          % Draw axes
          \draw[->, line width = 0.75pt] (0,0,0) -- (2.2,0,0) node[below]{$x$};
          \draw[->, line width = 0.75pt] (0,0,0) -- (0,2.2,0) node[above]{$y$};
          \draw[->, line width = 0.75pt] (0,0,0) -- (0,0,2.2) node[above]{$z$};

          


          % Draw lines connecting points
          \draw[myred, line width = 0.75pt] (a) -- (d);
          \draw[myred, line width = 0.75pt] (b) -- (d);
          \draw[myred, line width = 0.75pt] (b) -- (a);


          % Draw plane
          \fill[blue, opacity=0.3] (d) -- (a) -- (b) -- (c) -- cycle;
          % Draw points
          \foreach \point/\position in {d/above,a/below,b/below,c/right}
          {
            \draw[fill=black] (\point) circle (1pt) node [\position=3pt] {$\point$};
          }

          % Label plane
          % \node[blue, above right] at (B) {};
        \end{tikzpicture}
        \caption{3D s polytopem dimenze 2}
      \end{subfigure}
      \begin{subfigure}{0.4\textwidth}
        \begin{tikzpicture}[scale=2,tdplot_main_coords]
          % Define coordinates
          \coordinate (a) at (0.5,0.5,0);
          \coordinate (b) at (1.5,0.5,0);
          \coordinate (c) at (1.5,1.25,1);
          \coordinate (d) at (0.5,1.5,0);

          % Draw axes
          \draw[->, line width = 0.75pt] (0,0,0) -- (2.2,0,0) node[below]{$x$};
          \draw[->, line width = 0.75pt] (0,0,0) -- (0,2.2,0) node[above]{$y$};
          \draw[->, line width = 0.75pt] (0,0,0) -- (0,0,2.2) node[above]{$z$};



          % Draw lines connecting points
          \draw[myred, line width = 0.75pt] (a) -- (d);
          \draw[myred, line width = 0.75pt, dotted] (b) -- (d);
          \draw[myred, line width = 0.75pt] (b) -- (a);
          \draw[myred, line width = 0.75pt] (a) -- (c);
          \draw[myred, line width = 0.75pt] (b) -- (c);
          \draw[myred, line width = 0.75pt] (d) -- (c);
          % Draw plane
          \fill[blue, opacity=0.3] (d) -- (a) -- (b) -- (c) -- cycle;
          % Draw points
          \foreach \point/\position in {d/above,a/below,b/below,c/right}
            {
              \draw[fill=black] (\point) circle (1pt) node [\position=3pt] {$\point$};
            }
        \end{tikzpicture}
        \caption{3D s polytopem dimenze 3}
      \end{subfigure}
    \end{figure}
  \end{otazka}
\end{frame}
% \subsection{Otázka 3}
% \begin{frame}
%   \begin{otazka}{Co má větší vliv na časovou náročnost algoritmu? Dimenze či počet bodů? \\V jakém smyslu a proč?}
%     \begin{itemize}[label=\textbullet]
%       \item Pokud sortíme polytopy podle jejich obvodu:
%       \begin{itemize}[label=\textbullet]
%         \item Best: $\binom{\#V}{n+1}+\binom{n+1}{2}\binom{\#V}{n+1}+n \log n+n^3$
%         \item Worst:$\binom{\#V}{n+1}+\binom{n+1}{2}\binom{\#V}{n+1}+n \log n+ \binom{\#V}{n+1}n^3$: všechny polytopy + obvod pro všechny polytopy + sorting + determinant pro všechny polytopy
%       \end{itemize}
%       \item Pokud vybíráme minimální:
%       \begin{itemize}[label=\textbullet]
%         \item Best: $\binom{\#V}{n+1}+\binom{n+1}{2}\binom{\#V}{n+1}+n \cdot n^3$
%         \item Worst:$\binom{\#V}{n+1}+\binom{n+1}{2}\binom{\#V}{n+1}+n\binom{\#V}{n+1}n^3$: všechny polytopy + obvod pro všechny polytopy + determinant pro všechny polytopy ale po každé vybírám ten s minimálním obvodem (to je to $n$ před $\#VC(n+1)$)
%       \end{itemize}
%     \end{itemize}
%   \end{otazka}
% \end{frame}
\begin{frame}
  \begin{otazka}{Co má větší vliv na časovou náročnost algoritmu? Dimenze či počet bodů? \\V jakém smyslu a proč?}
    \begin{itemize}[label=\textbullet]
        \item Ani jedno, vzájemně se doplňují.
        \item Pro fixní dimenzi se zvyšováním počtu bodů algoritmus výrazně zpomaluje.
        \item Pro fixní počet bodů je algoritmus:
        \begin{itemize}
          \item nejrychlejší, když $n = 1,m - 1 $, a
          \item nejpomalejší, když $n = \lfloor m/2\rfloor$.
        \end{itemize}
    \end{itemize}
  \end{otazka}
\end{frame}
% \subsection{Otázka 4}
\begin{frame}
  \begin{otazka}{Můžeš odhadnout náročnost tvého algoritmu pro $n=10$ (alespoň přibližně)?}
  \begin{minipage}[t]{0.39\textwidth}
    \begin{itemize}[label=\textbullet]
      \item Časová náročnost algoritmu v~10D s náhodně generovanými body:
      \begin{itemize}[label=\textbullet]
        \item $\mathcal{O}(m^{11}\cdot \log m)$, kde $m$ je počet bodů.
      \end{itemize}
    \end{itemize}
  \end{minipage}
  \hfill
  \begin{minipage}[t]{0.6\textwidth}
    \begin{figure}[H]
      \centering
      \begin{tikzpicture}[scale=0.7]
          \begin{axis}[
              xlabel={počet bodů},
              ylabel={čas v sekundách},
              legend pos=north west,
                grid=both,
                xtick={10,12, 14, 16, 18, 20, 22, 24, 26, 28},
                ytick={0, 100, 200, 300, 400, 500, 600, 700, 800},
              ]
          % Function 4
          \addplot[color=purple, mark=*] coordinates {
            (10, 0.0003)
            (11, 0.0003)
            (12, 0.0004)
            (13, 0.0016)
            (14, 0.0039)
            (15, 0.0137)
            (16, 0.0436)
            (17, 0.1290)
            (18, 0.3331)
            (19, 0.7893)
            (20, 1.8222)
            (21, 4.1202)
            (22, 8.0600)
            (23, 15.826)
            (24, 29.858)
            (25, 54.231)
            (26, 105.23)
            (27, 719.52)
          };
          
          \legend{Časová náročnost algoritmu v 10D.}
          \end{axis}
      \end{tikzpicture}
      % \caption*{ Časová náročnost algoritmu v 10D.}
      \label{obr:graphs}
      \end{figure}
  \end{minipage}
  
  \end{otazka}
  
\end{frame}
% \begin{frame}
%   \frametitle{Děkuji za pozornost}
% \end{frame}
\end{document}
