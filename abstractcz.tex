\subsection*{Abstrakt}

Cílem práce je najít způsob, jak nalézt polytop maximální dimenze s minimálním obvodem z~množiny bodů v libovolné dimenzi. Tento matematický problém rozdělíme do tří variant: problém v 1D, ve 2D, a obecně v $n$D.

V 1D budeme hledat minimální vzdálenost mezi dvěma po sobě jdoucími čísly v seřazené množiny bodů. Ve 2D úlohu převedeme na graf, ve kterém budeme hledat cyklus s minimální váhou a délkou tři. Lehce nám to zkomplikuje kontrolování, jestli polytop má maximální dimenzi. Proto musíme ověřit, zda body neleží na přímce. V $n$D budu hledat polytop tvořený $n+1$ body. Zde bude ale mnohem těžší zkontrolovat, jestli polytop má maximální dimenzi.

V praktické části naprogramuji algoritmy z teoretické části, změřím kolik času potřebují v závislosti počtu bodů, a nakonec je mezi sebou porovnám. 

% Zde jsou klíčová slova. Přepisujte v data.tex!
\vspace{1em}
\noindent
\textbf{Klíčová slova: }\keywordscz

% Zde jsou použité zkratky. Odkomentujte, pokud nutno.
% \vspace{1em}
% \noindent
% \textbf{Použité zkratky: } BLA -- bla, bla, bla
