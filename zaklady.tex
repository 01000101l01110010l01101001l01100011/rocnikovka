\chapter*{Základní definice}
\label{sec:zakladni-pojmy}

\begin{definition}[Polytop]
    \emph{Polytop} dimenze $n \in \mathbb{N}$ je uzavřená podmnožina $P \subseteq \mathbb{R}^{n}$ definovaná induktivně:
    \begin{itemize}
        \item Polytop dimenze $1$ je úsečka.
        \item Polytop dimenze $n$ je slepením polytopů dimenze $n-1$, jež spolu mohou
            sdílet stěny libovolné dimenze, kde \emph{stěnou} polytopu rozumíme jeho
            libovolnou podmnožinu jsoucí rovněž polytopem. \autocite{adamklepacDefinicePolytopu2024}
    \end{itemize}
    
\end{definition}

\begin{definition}[Bod]
    \label{definice:bod}
    Bod je uspořádaná $n$-tice $(x_1, x_2, \ldots, x_n) \in \mathbb{R}^n$. Ve $2D$ budu používat značení $a=(a_x, a_y)$
\end{definition}

\begin{definition}[Vzdálenost]
    \label{definice:vzdalenost}
    %TODO je to true to R^nxR^n???
    Zobrazení $d: \mathbb{R}^n\times \mathbb{R}^n~\rightarrow~\mathbb{R}^+$ nám určí vzdálenost dvou bodů $u, v \in \mathbb{R}^n$
    podle předpisu $d(u, v) \coloneqq \sqrt{(v_1-u_1)^2+(v_2-u_2)^2+\ldots+(v_n-u_n)^2}$.
\end{definition}

\begin{definition}[Ohodnocený graf]
    \label{definice:ohodnoceny_graf}
    $G = (V, E, w)$ je ohodnocený graf, kde $V$ je množina vrcholů, $E$ je množina dvojprvkových podmnožin $E \subseteq \binom{V}{2}$ a $w$ je libovolné zobrazení $E \rightarrow \mathds{R}^+$, které hranám přiřazuje jejich váhu.
\end{definition}

\begin{definition}[Úplný ohodnocený graf]
    \label{definice:uplny_ohodnoceny_graf}
    Úplný ohodnocený graf $G = (V, E, w)$ má kaž\-dé dva vrcholy spojeny hranou, neboli $E = \binom{V}{2}$. Takový graf můžeme také zapsat jako $K_n \coloneqq (V,\binom{V}{2},w)$.
    \begin{figure}[h]
        \centering
        \begin{subfigure}[b]{0.3\textwidth}
            \centering
            \begin{tikzpicture}
                \graph { subgraph K_n [n=3, clockwise, radius=0.3\textwidth, nodes={font=\small}] };
            \end{tikzpicture}
        \end{subfigure}
        \begin{subfigure}[b]{0.3\textwidth}
            \centering
            \begin{tikzpicture}
                \graph { subgraph K_n [n=5, clockwise, radius=0.3\textwidth, nodes={font=\small}] };
            \end{tikzpicture}
        \end{subfigure}
        \begin{subfigure}[b]{0.3\textwidth}
            \centering
            \begin{tikzpicture}
                \graph { subgraph K_n [n=7, clockwise, radius=0.3\textwidth, nodes={font=\small}] };
            \end{tikzpicture}
        \end{subfigure}
        \caption{Úplné grafy $K_3$, $K_5$ a $K_7$}
        \label{obr:uplne_ohodnocene_grafy}

    \end{figure}
\end{definition}

\begin{definition}[Podgraf]
    Podgraf $H = (V, E)$ je tvořen podmnožinou vrcholů a hran jiného grafu. Hrany grafu $H$ musí mít oba vrcholy v množině vrcholů $V$.
    
\end{definition}

\begin{definition}[Cesta]
    \label{definice:cesta}
    Cestou v grafu nazveme posloupnost \textbf{různých} vrcholů $v_1, \ldots, v_n$, pokud $\forall i \in \{1,\ldots, n-1\}$ platí $\{v_i, v_{i+1}\} \in E$.  
\end{definition}

\begin{definition}[Váha cesty]
    \label{definice:vaha_cesty}
    Pokud cestu tvoří posloupnost vrcholů $v_1, \ldots, v_n$, tak váha cesty je rovna \[ w(\{v_0,\dots ,v_n\}) = \sum_{i=1}^{n-1}w(\{v_i, v_{i+1}\}). \]
    
\end{definition}

\begin{definition}[Cyklus]
    \label{definice:cyklus}
    Cyklus je posloupnost vrcholů $v_1,\ldots,v_n,v_1$, kde $v_1,\ldots,v_n$ je cesta a poslední dva vrcholy $\{v_1,v_n\} \in E$.
\end{definition}

\begin{definition}[Váha cyklu]
    \label{definice:vaha_cyklu}
    Pokud cyklus tvoří posloupnost vrcholů $v_1, \ldots, v_n$, tak váha \\cesty je rovna \[ w(\{v_0,\dots ,v_n, v_0\}) = w(\{v_1, v_n\}) + \sum_{i=1}^{n-1}w(\{v_i, v_{i+1}\}). \]
    
\end{definition}

\begin{definition}[Trojúhelníková nerovnost]
    \label{definice:trojuhelnikova_nerovnost}
    Trojúhelníková nerovnost říká, že pro kaž\-dé tři různé body $a, b, c$ platí \textcolor{myblue}{$d(a, b)+d(c, b)$} $\geq$ \textcolor{myred}{$d(a,b)$}, neboli vzdálenost mezi dvěma body je vždy menší nebo rovna součtu vzdáleností mezi těmito body a třetím bodem. 

    \begin{figure}[h]
        \centering
        \begin{tikzpicture}
            \label{obr:trojuhelnikova_nerovnost}
            \coordinate[label=below left:$a$] (A) at (0,0);
            \coordinate[label=below right:$b$] (B) at (6,0);
            \coordinate[label=above:$c$] (C) at (3,2);
            \draw[myred, line width=1.5pt] (A) -- (B);
            \draw[myblue, line width=1.5pt] (B) -- (C);
            \draw[myblue, line width=1.5pt] (C) -- (A);
            \fill (A) circle[radius=2pt];
            \fill (B) circle[radius=2pt];
            \fill (C) circle[radius=2pt];
        \end{tikzpicture}
        \caption{Trojúhelníková nerovnost}
    \end{figure}

\end{definition}
\begin{definition}[Soused]
    \label{definice:soused}
    V grafu $G = (V, E, w)$ je bod $u$ soused bodu $v$, pokud $\{u, v\} \in E$.
\end{definition}

\begin{definition}[Big $O$ notation]
    \label{definice:bigonotation}
    idk
\end{definition}

\begin{definition}[Obvod trojúhelníku]
    \label{definice:obvod_troj}
    Definujeme si zobrazení $p: E\rightarrow \mathbb{R}^+$ podle předpisu: $p(a, b, c) \coloneqq d(a, b) + d(b, c) + d(a, c)$, které nám určuje obvod trojúhelníku $a, b, c$.
\end{definition}