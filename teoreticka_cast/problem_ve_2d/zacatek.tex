\section{Začátek}
\label{sec:zacatek}

% Tohle změním někdy až potom. Napíšu nějaký lepší úvod
\textcolor{myred}{Tohle upravím až potom, já jsem to napsal do latexu, ale to sis asi nepřečetl. Chtěl bych napsat lepší a delší úvod.}

Úkolem je nalézt trojúhelník minimálního obvodu s vrcholy v zadané množině rovinných bodů. Ano, mohli bychom zkoušet všechny trojúhelníky, ale to by zabralo až moc času. Pokusím se tedy najít nějaký algoritmus, který by to zvládl rychleji.

Nejlepší způsob jak řešit úlohy tohoto typu je převést si úlohu na grafovou úlohu. V teorii grafů nás nezajímá umístění bodů a hran, ale zajímá nás pouze to, jak jsou tyto body a hrany propojeny. Hrany pak mají svoje ohodnocení, neboli váhu. V našem případě bude váha hrany odpovídat délce úsečky, která spojuje dva body.