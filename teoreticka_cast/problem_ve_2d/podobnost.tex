\section{Podobnost trojúhelníků}
\label{sec:podobnost}
V \myref{sekci}{sec:algoritmus} Vám představím algoritmus na hledání trojúhelníku s minimálním obvodem. Velmi důležitou částí tohoto algoritmu je kontrolování, jestli body tvoří trojúhelník, to znamená, jestli body neleží na přímce. K tomuto máme následující tvrzení.

\begin{tvrzeni}[Podobnost trojúhelníků a body na jedné přímce]
    \label{tvrzeni:podobnost_trojuhelniku}
    Jsou dány 3 body v $\mathbb{R}^2$: $a = (a_x, a_y)$, $b = (b_x, b_y)$, $c = (c_x, c_y)$. Tyto body jsou na přímce, pokud platí, že trojúhelníky $\Delta((a_x, a_y), (b_x, a_y), (b_x, b_y))$~a~$\Delta((a_x, a_y), (c_x, a_y), (c_x, c_y))$ mají stejný poměr stran, proto platí rovnice: $(\textcolor{myred}{c_y}-\textcolor{myred}{a_y})(\textcolor{myblue}{b_x}-\textcolor{myblue}{a_x}) = (\textcolor{myorange}{b_y}-\textcolor{myorange}{a_y})(\textcolor{mygreen}{c_x}-\textcolor{myblue}{a_x})$.

    \begin{figure}[h]
        \label{obr:podobnost_trojuhelniku}
    \centering
    \begin{tikzpicture}
        \draw[->, line width=1pt] (4.5,0) -- (14,0) node[right] {$x$};
        \draw[->, line width=1pt] (5,-0.5) -- (5,3.5) node[above] {$y$};

        \fill (5, 1) circle[radius=2pt];
        \node at (5, 1) [left] {1};
        \fill (5, 2) circle[radius=2pt];
        \node at (5, 2) [left] {2};
        \fill (5, 3) circle[radius=2pt];
        \node at (5, 3) [left] {3};

        \fill (6, 0) circle[radius=2pt];
        \node at (6, 0) [below] {6};
        \fill (9, 0) circle[radius=2pt];
        \node at (9, 0) [below] {9};
        \fill (12, 0) circle[radius=2pt];
        \node at (12, 0) [below] {12};
        

        \coordinate[label=above left:$a$] (A) at (6,1);
        \coordinate[label=above left:$b$] (B) at (9,2);
        \coordinate[label=above left:$c$] (C) at (12,3);

        \coordinate (D) at (9, 1);
        \coordinate (E) at (12, 1);
        \draw[line width=1.5pt] (A) -- (B);
        \draw[line width=1.5pt] (B) -- (C);
        \draw[myblue, line width=1.5pt] (A) to node[midway, below] {$3$} (D);
        \draw[mygreen, line width=1.5pt] (D) to node[midway, below] {$3$} (E);
        \draw[myred, line width=1.5pt] (C) to node[midway, right] {$2$} (E);
        \draw[myorange, line width=1.5pt] (B) to node[midway, right] {$1$} (D);
        \fill (A) circle[radius=2pt];
        \fill (B) circle[radius=2pt];
        \fill (C) circle[radius=2pt];

    \end{tikzpicture}
    \caption{Příklad podobnosti trojúhelníků}
    \end{figure}
\end{tvrzeni}

\begin{proof}
    Předpokládejme, že existuje lineární funkce $\mathfrak{f}: y = mx+k$, na které leží všechny tři body. Po dosazení bodů do rovnice nám vznikne soustava rovnic o třech neznámých. 
    \begin{align*}
        a_y &= ma_x+k \\
        b_y &= mb_x+k \\
        c_y &= mc_x+k 
    \end{align*}
    Poté odečteme třetí od první a první od druhé. Tím dostaneme:
    \begin{align*}
        a_y-c_y &= ma_x+k - mc_x -k \\
        b_y-a_y &= mb_x+k -ma_x -k 
    \end{align*}
    Vytknutím $m$ dostaneme:
    \begin{align*}
        a_y-c_y &= m(a_x-c_x) \\
        b_y-a_y &= m(b_x-a_x) 
    \end{align*}
    A tím pádem $m_1 = (a_y-c_y)/(a_x-c_x)$ a $ m_2=(b_y-a_y)/(b_x-a_x)$. My ale víme, že $m_1$ a $m_2$ jsou stejná čísla, protože funkce $\mathfrak{f}$ protíná všechny tři body. Proto můžeme sestavit rovnici $(a_y-c_y)/(a_x-c_x) = (b_y-a_y)/(b_x-a_x)$. Nakonec můžeme rovnici upravit do tvaru $(a_y-c_y)(b_x-a_x) = (b_y-a_y)(a_x-c_x)$. Tímto je \myref{tvrzení}{tvrzeni:podobnost_trojuhelniku} dokázáno.
\end{proof}