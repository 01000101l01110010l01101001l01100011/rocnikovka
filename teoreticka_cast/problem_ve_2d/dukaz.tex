\section{Důkaz algoritmu}
\label{dukaz_algoritmu}

Abychom mohli dokázat korektnost algoritmu, musíme dokázat, že algoritmus skončí a že je správný, to znamená, že dělá přesně co chceme. Dijkstrův algoritmus, který je součástí algoritmu, dokazovat nebudu, protože důkaz je příliš dlouhý a je dostupný v literatuře.

Dokázat konečnost algoritmu není těžké. Jediným cyklem v algoritmu je procházení všech stran. Jelikož je množina hran konečná, tento cyklus skončí.  Dále používáme Dijkstrův algoritmus, který také skončí, protože náš graf je souvislý. 

Dokázat, že algoritmus funguje je těžší. Povedeme důkaz sporem; náš předpoklad bude následující výrok:

\begin{vyrok}[Předpoklad]
Existuje trojúhelník $x, y, z$, který má kratší obvod, než trojúhelník $a, b, c$, který našel algoritmus, neboli:
\begin{equation*}
    \exists \{x, y, z\}\subseteq V: d(x, y) + d(y, z) + d(x, z) < d(a, b) + d(b, c) + (a, c)| \forall a, b, c \leftarrow algoritmus.
\end{equation*}
    % Ještě nevím, jestli tam nechám tu část po \forall, nevím jestli to takhle dává dobrý smysl.
\end{vyrok}

