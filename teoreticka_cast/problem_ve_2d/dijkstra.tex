\section{Adaptace Dijkstrova algoritmu}
\label{sec:dijkstra}

Velmi důležitou části algoritmu je Dijkstrův Algoritmus, který v ohodnoceném grafu dokáže najít cestu s minimální váhou mezi dvěma body v čase $ \mathcal{O}(\#V^2)$. \autocite{benlFormalCorrectnessProofs1999} Obecně taková cesta může mít několik vrcholů, ale protože náš graf je převzatý z~roviny, povede právě přes jeden vrchol a bude se skládat ze dvou hran. To vyplývá z \hyperref[definice:trojuhelnikova_nerovnost]{trojúhelníkové nerovnosti}. Kdyby náš graf nebyl převzatý z~roviny, mohla nastat situace na \myref{obrázku}{obr:troj_ner_graf}. Dijkstrův algoritmus ale lehce upravíme, aby byl rychlejší. Jelikož Dijkstrův algoritmus neví, že náš graf je převzatý z roviny (to znamená, že v grafu platí \hyperref[definice:trojuhelnikova_nerovnost]{trojúhelníková nerovnost}), tak ho ukončíme dřív, aby nehledal cestu s minimální váhou přes více vrcholů. Tímto časovou náročnost zredukujeme na $ \mathcal{O}(\#V)$.

\begin{figure}[H]
  \centering
  \begin{tikzpicture}[node distance=2cm]
    \node[draw, circle, fill, inner sep=1.5pt] (A) at (0,0) {};
    \node[below] at (A.south) {$u$};
    \node[draw, circle, fill, inner sep=1.5pt] (B) at (2,1.5) {};
    \node[below] at (B.south) {$j$};
    \node[draw, circle, fill, inner sep=1.5pt] (D) at (6,1.5) {};
    \node[below] at (D.south) {$v$};
    \node[draw, circle, fill, inner sep=1.5pt] (C) at (-2,1.5) {};
    \node[below] at (C.south) {$k$};
    \node[draw, circle, fill, inner sep=1.5pt] (E) at (8,0) {};
    \node[below] at (E.south) {$x$};

    \draw[myblue, line width=1.5pt] (A) -- node[above] {2} (B);
    \draw[myblue, line width=1.5pt] (B) -- node[above] {3} (D);
    \draw[myred, line width=1.5pt] (A) -- node[below] {7} (D);
    \draw[mygreen, line width=1.5pt] (A) -- node[below] {3} (C);
    \draw[mygreen, line width=1.5pt] (D) -- node[above] {7} (E);
  \end{tikzpicture}
  \caption{V obecném grafu nemusí platit \hyperref[definice:trojuhelnikova_nerovnost]{trojúhelníková nerovnost}.}
  \label{obr:troj_ner_graf}
\end{figure}

\subsection{Popis adaptace Dijkstrova algoritmu}
\label{subsec:dijkstra_adaptace}
  Vstupem je bod $a$ a cílem je bod $b$. Cestu s minimální váhou mezi těmito body můžeme najít všechny cesty, které vedou z $a$ do $b$ a vybrat tu nejkratší. V \myref{algoritmu}{alg:algoritmus} je adaptace Dijkstrova algoritmu funkce $path$.

\begin{figure}[H]
  \centering
  \begin{tikzpicture}[scale=0.1]
    \graph { subgraph K_n [n=5, clockwise, radius=0.1\textwidth, nodes={font=\small}] };
  \end{tikzpicture}
  \caption{Graf $K_5$ na kterém si můžete představit jak by tento algoritmus fungoval.}
\end{figure}

\subsection{Popis Dijkstrova algoritmu}
\label{subsec:popis-algoritmu}
Dijkstrův algoritmus počítá také nejkratší vzdálenosti z bodu $a$ do všech ostatních. Tyto vzdálenosti by teoreticky šly využít na zrychlení celého algoritmu, ale je otázkou co by bylo rychlejší: nechat doběhnout celý Dijkstrův algoritmus a pak z něho brát informace nebo využít adaptaci Dijkstrova algoritmu, která je rychlejší. Následuje popis Dijkstrova algoritmu.

\begin{enumerate}
  \item Vytvoříme množinu všech nenavštívených bodů a vybereme startovní a cílový bod. Označíme všechny body nenavštívenými.
  \item Každému bodu přiřadíme vzdálenost od počátečního bodu; prozatím na $\infty$. Vzdálenost počátečního bodu od sebe samého nastavíme na $0$.
        % \item Nejdříve projdeme všechny sousedy počátečního bodu. Každému sousedu spočítáme vzdálenost od počátečního bodu (v tomto případě váha hrany vedoucí z~počátečního bodu do souseda) a připíšeme ji sousedovi, bude-li menší než $\infty$. Až zkontrolujeme všechny jeho sousedy, odebereme počáteční bod z množiny nenavštívených bodů.
  \item Přesuneme na nenavštívený bod s minimální vzdáleností od počátečního bodu (poprvé to bude počáteční bod).
        Tento bod označíme jako aktuální a začneme kontrolovat jeho sousedy.
        Je-li součet vzdálenosti od počátečního bodu do aktuálního s váhou hrany vedoucí k sousedu menší než vzdálenost, kterou má u sebe soused uloženou, změníme ji.
        Je třeba myslet na to, že když přepíšeme vzdálenost souseda od startovního bodu, souseda neoznačujeme za navštíveného. Počáteční bod odebereme z množiny nenavštívených bodů, až zkontrolujeme všechny jeho sousedy.
  \item Čtvrtý bod opakujeme, dokud nevybereme za aktuální bod ten cílový. V tomto okamžiku jsme našli nejkratší cestu. Pokud bychom chtěli získat nejkratší vzdálenosti z počátečního bodu do všech ostatních, čtvrtý bod budeme opakovat dokud nebude množina nenavštívených bodů prázdná.
\end{enumerate}