\section{Dijkstrův algoritmus}
\label{sec:dijkstra}

Dijkstrův algoritmus, pojmenovaný po Edsger W. Dijkstrovi, je algoritmus na hledání nejkratší cesty mezi dvěma body v ohodnoceném grafu. 

\subsection{Popis algoritmu}
\label{subsec:popis-algoritmu}

\begin{enumerate}
    \item Vytvoříme si množinu všech nenavštívených bodů a vybereme si startovní a cílový bod. Nastavíme všechny body jako nenavštívené. 
    \item Ke každému bodu přiřadíme vzdálenost od počátečního bodu. Prozatím tuto vzdálenost u všech bodů nastavíme na $\infty$. Vzdálenost počátečního bodu od sebe samého nastavíme na $0$.
    \item Vybereme si aktuální bod takový, že je nenavštívený a má nejmenší vzdálenost od počátečního bodu. Začneme porovnávat všechny jeho nenavštívené sousedy. Budeme je porovnávat v pořadí ve kterém jsou sousedi nejblíž aktuálnímu bodu. Zjistíme jak daleko je aktuální bod daleko od startovního bodu a jak daleko je soused od aktuálního bodu. Tyto dvě vzdálenosti sečteme a pokud výsledná vzdálenost je menší, než ta uložená u souseda, tak ji přepíšeme. Tento postup zopakujeme pro všechny nenavštívené sousedy aktuálního bodu.
    \item Když zkontrolujeme všechny nenavštívené sousedy, můžeme aktuální bod odeberat z množiny nenavštívených bodů. Potom se přesuneme na nenavštívený bod s nejmenší vzdáleností od počátečního bodu a opakujeme bod 3. Pozor na to, že když přepíšu vzdálenost souseda od startovního bodu, souseda nenavštěvuji.
    \item Pokud je aktuální bod cílový bod, tak jsme našli nejkratší cestu.
\end{enumerate}