\section{Adaptace Dijkstrova algoritmu}
\label{sec:dijkstra}

Dijkstrův algoritmus, pojmenovaný po Edsgerovi W. Dijkstrovi, je algoritmus na hledání cesty s minimální váhou mezi dvěma body v ohodnoceném grafu. Obecně taková cesta může mít několik vrcholů, ale protože náš graf je úplný a převzatý z roviny, tak cesta s minimální váhou mezi vrcholy $u$ a $v$ je hrana $\{u, v\}$. Pozor! Toto funguje pouze protože náš graf je převzatý z roviny. Situace na \myref{obrázku}{obr:troj_ner_graf} by v rovině a našem grafu nemohla nastat kvůli \hyperref[definice:trojuhelnikova_nerovnost]{trojúhelníkové nerovnosti}. Aby nám vznikl trojúhelník, tak hranu $\{u, v\}$ z grafu odebereme a až teď budeme hledat nejkratší cestu mezi body $u$ a $v$. Nejkratší cesta povede právě přes jeden vrchol (plyne z \hyperref[definice:trojuhelnikova_nerovnost]{trojúhelníkové nerovnosti}), který nazveme $j$, a bude se skládat právě ze dvou hran. Jedna hrana bude vést z $u$ do $j$ a druhá z $j$ do $v$. 



\begin{figure}[h]

    \centering
    \begin{tikzpicture}[node distance=2cm]
        \node[draw, circle, fill, inner sep=1.5pt] (A) at (0,0) {};
        \node[below] at (A.south) {$u$};
        \node[draw, circle, fill, inner sep=1.5pt] (B) at (2,2) {};
        \node[below] at (B.south) {$j$};
        \node[draw, circle, fill, inner sep=1.5pt] (D) at (6,2) {};
        \node[below] at (D.south) {$v$};
        \node[draw, circle, fill, inner sep=1.5pt] (C) at (-2,2) {};
        \node[below] at (C.south) {$k$};
        \node[draw, circle, fill, inner sep=1.5pt] (E) at (8,0) {};
        \node[below] at (E.south) {$x$};
        
        \draw[myblue, line width=1.5pt] (A) -- node[above] {2} (B);
        \draw[myblue, line width=1.5pt] (B) -- node[above] {3} (D);
        \draw[myred, line width=1.5pt] (A) -- node[below] {7} (D);
        \draw[mygreen, line width=1.5pt] (A) -- node[below] {3} (C);
        \draw[mygreen, line width=1.5pt] (D) -- node[above] {7} (E);
    \end{tikzpicture}
    \caption{V obecném grafu nemusí fungovat trojúhelníková nerovnost.}
    \label{obr:troj_ner_graf}
\end{figure}




\subsection{Popis algoritmu}
\label{subsec:popis-algoritmu}

\begin{enumerate}
    \item Vytvoříme si množinu všech nenavštívených bodů a vybereme si startovní a cílový bod. Označíme všechny body nenavštívenými. 
    \item Každému bodu přiřadíme vzdálenost od počátečního bodu; prozatím na $\infty$. Vzdálenost počátečního bodu od sebe samého nastavíme na $0$. 
    \item Nejdříve projdeme všechny sousedy počátečního bodu. Každému sousedovi spočítáme vzdálenost od počátečního bodu (v tomto případě váha hrany vedoucí z počátečního bodu do souseda) a připíšeme ji sousedovi, bude-li menší než $\infty$. Až zkontrolujeme všechny jeho sousedy, můžeme počáteční bod odebrat z množiny nenavštívených bodů.
    \item Potom se přesuneme na nenavštívený bod s minimální vzdáleností od počátečního bodu. 
    Tento bod označíme jako aktuální a začneme kontrolovat jeho sousedy. Budeme se dívat, jestli vzdálenost od počátečního bodu do aktuálního + váha hrany vedoucí k sousedovi nejsou menší než vzdálenost, kterou má soused u sebe uloženou. Pokud výsledná hodnota bude menší než ta, kterou má u sebe uloženou soused, změníme ji. Je třeba myslet na to, že když přepíšeme vzdálenost souseda od startovního bodu, souseda nenavštěvujeme. Počáteční bod můžeme odebrat z množiny nenavštívených bodů až zkontrolujeme všechny jeho sousedy.
    \item Čtvrtý bod opakujeme, dokud nevybereme za aktuální bod ten cílový. V tomto okamžiku jsme našli nejkratší cestu.
\end{enumerate}