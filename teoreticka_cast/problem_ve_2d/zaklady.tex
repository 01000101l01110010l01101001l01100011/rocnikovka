\section{Základní pojmy}
\label{sec:zakladni-pojmy}

\begin{definition}[Vzdálenost]
    \label{definice:vzdalenost}
    Vzdálenost mezi dvěma body $u$ a $v$ nám určuje zobrazení \\$d(u, v)~\rightarrow~\mathds{R}^+$ s předpisem: $d(u, v) = \sqrt{(v_x-u_x)^2+(v_y-u_y)^2}$, kde $v_x$ a $v_y$ jsou souřadnice bodu $v$. 
\end{definition}

\begin{definition}[Ohodnocený graf]
    \label{definice:ohodnoceny_graf}
    $G = (V, E, w)$ je ohodnocený graf, kde $V$ je množina vrcholů, $E$ je dvojprvková podmnožina $E \subseteq \binom{V}{2}$ a $w$ je libovolné zobrazení $E \rightarrow \mathds{R}^+$, které hranám přiřazuje jejich váhu.
\end{definition}

\begin{definition}[Úplný ohodnocený graf]
    \label{definice:uplny_ohodnoceny_graf}
    Úplný ohodnocený graf $G = (V, E, w)$ má každé dva vrcholy spojeny hranou, neboli $E = \binom{V}{2}$. Takový graf můžeme také zapsat jako $K_n = (V, \binom{V}{2}, w)$.
    \begin{figure}[h]
        \centering
        \begin{subfigure}[b]{0.3\textwidth}
            \centering
            \begin{tikzpicture}
                \graph { subgraph K_n [n=3, clockwise, radius=0.3\textwidth, nodes={font=\small}] };
            \end{tikzpicture}
        \end{subfigure}
        \begin{subfigure}[b]{0.3\textwidth}
            \centering
            \begin{tikzpicture}
                \graph { subgraph K_n [n=5, clockwise, radius=0.3\textwidth, nodes={font=\small}] };
            \end{tikzpicture}
        \end{subfigure}
        \begin{subfigure}[b]{0.3\textwidth}
            \centering
            \begin{tikzpicture}
                \graph { subgraph K_n [n=7, clockwise, radius=0.3\textwidth, nodes={font=\small}] };
            \end{tikzpicture}
        \end{subfigure}
        \caption{Úplné grafy $K_3$, $K_5$ a $K_7$}
        \label{obr:uplne_ohodnocene_grafy}

    \end{figure}
\end{definition}

\begin{definition}[Cesta]
    \label{definice:cesta}
    Cestou v grafu nazveme posloupnost \textbf{různých} vrcholů $v_1, \dots, v_n$, pokud $\forall i \in \{1,\dots, n-1\}$ platí $\{v_i, v_{i+1}\} \in E$.  
\end{definition}

\begin{definition}[Váha cesty]
    \label{definice:vaha_cesty}
    Pokud cestu tvoří posloupnost vrcholů $v_1, \dots, v_n$, tak váha cesty bude rovna \[ \sum_{i=1}^{n-1}w(\{v_i, v_{i+1}\}) \]
    
\end{definition}

\begin{definition}[Cyklus]
    \label{definice:cyklus}
    Cyklus tvoří cesta z bodu $u$ do bodu $v$ a koncový bod $u$.    
\end{definition}

\begin{definition}[Váha cyklu]
    \label{definice:vaha_cyklu}
    Pokud cyklus tvoří posloupnost vrcholů $v_1, \dots, v_n$, tak váha \\cesty bude rovna \[ w(\{v_1, v_n\}) + \sum_{i=1}^{n-1}w(\{v_i, v_{i+1}\}) \]
    
\end{definition}

\begin{definition}[Trojúhelníková nerovnost]
    \label{definice:trojuhelnikova_nerovnost}
    Trojúhelníková nerovnost říká, že pro každé tři různé body $A, B, C$ platí \textcolor{myblue}{$d(A, B)+d(C, B)$} $\geq$ \textcolor{myred}{$d(A,B)$}, neboli vzdálenost mezi dvěma body je vždy menší nebo rovna součtu vzdáleností mezi těmito body a třetím bodem. 

    \begin{figure}[h]
        \centering
        \begin{tikzpicture}
            \label{obr:trojuhelnikova_nerovnost}
            \coordinate[label=below left:$A$] (A) at (0,0);
            \coordinate[label=below right:$B$] (B) at (6,0);
            \coordinate[label=above:$C$] (C) at (3,2);
            \draw[myred, line width=1.5pt] (A) -- node[below] {c} (B);
            \draw[myblue, line width=1.5pt] (B) -- node[right=10pt] {a} (C);
            \draw[myblue, line width=1.5pt] (C) -- node[left=10pt] {b} (A);
            \fill (A) circle[radius=2pt];
            \fill (B) circle[radius=2pt];
            \fill (C) circle[radius=2pt];
        \end{tikzpicture}
        \caption{Trojúhelníková nerovnost}
    \end{figure}

    \begin{definition}[Soused]
        \label{definice:soused}
        V grafu $G = (V, E, w)$ je bod $u$ soused bodu $v$, pokud $\{u, v\} \in E$.
    \end{definition}
    
\end{definition}