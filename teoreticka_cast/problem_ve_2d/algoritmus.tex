\section{Algoritmus}
\label{sec:algoritmus}

Nechť $V$ je množina bodů v rovině, pro každé dva body $u, v \in V$ označme $d(u, v)$ vzdálenost mezi nimi. Pomocí funkce $d$ definujeme graf $G = (V, E, w)$, kde hrany $E = \{\{u, v\} \mid u, v \in V\}$ a váha, neboli ohodnocení $w(\{u, v\}) = d(u, v)$. Nyní chceme najít trojúhelník. Náhodně si vybereme jednu hranu $\{u, v\} \in E$ a odebereme ji z množiny hran $E$. Nyní chceme najít nejkratší cestu (O tom co je cesta si můžete přečíst v Appendixu idk) mezi body $u, v$. K tomuto použijeme Dijkstrův algoritmus. Pokud nějaká cesta existuje, společně s hranou $\{u, v\}$ tvoří nějaký mnohoúhelník. Není těžké si rozmyslet, že tento útvar bude trojúhelník. Cesta se bude skládat ze dvou hran $\{u, j\}$ a $\{j, v\}$, které společně s hranou $\{u, v\}$ tvoří trojúhelník $T = (V_T, E_T, w)$, kde $V_T = \{u, j, v\}$, $E_T = \{\{u,j\}, \{j,v\}, \{u,v\}\}$ a váha $w = w(\{u, v\}) + w(\{u, j\}) + w(\{j, v\})$. Pokud je tato váha menší než nejmenší váha, kterou jsme doposud našli, uložíme si tento trojúhelník. Nezapomene znovu přidat hranu $\{u, v\}$ zpátky do množiny hran $E$. Tento postup opakujeme, dokud nevyčerpáme všechny hrany. Výsledkem bude trojúhelník s nejmenším obvodem.

\begin{algorithm}
    \caption{Algoritmus na hledání trojúhelníku s nejkratším obvodem.}
    \label{alg:algoritmus}
   
    \SetKwInOut{Input}{input}
    \SetKwInOut{Output}{output}
    \SetKw{KwReturn}{return}
   
    \Input{Množina bodů $V$ v rovině, kde každý bod je reprezentován jako dvojice souřadnic $(v_x, v_y)$}
    \Output{Trojúhelník $T = (V_T, E_T, w)$}
    \BlankLine
    \For{$u \in V$}{
        \For{$v \in V$}{
        $d(u, v) = \sqrt{(v_x-u_x)^2+(v_y-u_y)^2}$\;
        }
    }
    $ E = {{V}\choose{2}}$\;
    $ G = (V, E, w)$\;
    $min_T = \infty$\;
    \For{$\{u, v\} \in E$}{
        $E = E \setminus \{u, v\}$\;
        $d(u, j, v) = dijkstra(G, u, v)$\;
        \If{$d(u, j, v) + d(u, v) < min_T$}{
            $min_T = d(u, j, v) + d(u, v)$\;
            $V_T = \{u, j, v\}$\;
            $E_T = \{\{u,j\}, \{j,v\}, \{u,v\}\}$\;
            $T = (V_T, E_T, w)$\;
        }
        $E = E \cup \{u, v\}$\;
    }
    \KwReturn{$T$}\;
   \end{algorithm}