\section{Algoritmus}
\label{sec:algoritmus}
% Následující algoritmus je časově efektivnější, než zkoušení všech možností, ale pořád zabere \textcolor{red}{IDK} čas. 

Nechť $V$ je množina bodů v rovině a pro každé dva body $u, v \in V$ označme $d(u, v)$ jejich vzájemnou vzdálenost. Množinu hran označíme $E = \binom{V}{2}$ a váhu, neboli ohodnocení, nám určuje zobrazení $w$ dané předpisem $w(\{u, v\}) \coloneqq d(u, v)$ $\forall (u, v) \in E$. Nyní můžeme definovat graf $G = (V, E, w)$.
Tímto je příprava hotova. Následuje hledání cyklu délky tři s minimální váhou.



Náhodně vybereme jednu hranu $\{u, v\} \in E$ a odebereme ji z množiny hran $E$. V~grafu bez hrany $\{u, v\}$ potřebujeme najít cestu s minimální váhou mezi body $u$ a  $v$. K~tomuto použijeme \hyperref[sec:adaptace]{algoritmus na hledání nejkratší cesty}, který nám vrátí cestu délky dva. Cesta povede právě přes jeden vrchol, protože z \hyperref[tvrzeni:trojuhelnikova_nerovnost]{trojúhelníkové nerovnosti} je jasné, že pokud by cesta vedla přes více vrcholů, byla by delší. Bude skládat ze dvou hran: $\{u, j\}$ a $\{j, v\}$.

Teď musíme zkontrolovat, zda tato cesta s hranou $\{u, v\}$ opravdu tvoří trojúhelník, protože se může  stát, že body $u, j, v$ leží na jedné přímce. K tomu využijeme \hyperref[sec:podobnost]{podobnost trojúhelníků}, z které vyplývá, že body se nacházejí na přímce právě tehdy, pokud rovnice $(v_y - u_y)(j_x - u_x) = (j_y - u_y)(v_x - u_x)$ je pravdivá. 
%Definujeme si body $a \coloneqq min(\{u_x, j_x, v_x\})$, $c \coloneqq max(\{u_x, j_x, v_x\})$ a bod $b$ je zbývající. Pak platí $a_x < b_x < c_x$. Body se nacházejí na přímce právě tehdy, pokud rovnice $\frac{c_y - a_y}{c_x - a_x} = \frac{b_y - a_y}{b_x - a_x}$ je pravdivá.

\begin{poznamka}[jiný způsob]
Předchozí způsob funguje perfektně pro počítače, ale pro člověka je v některých případech zbytečně komplikovaný. Tyto případy nastávají, když se body nacházejí na horizontálních nebo vertikálních přímkách. To jde poznat tak, že $u_y = j_y = v_y$ nebo $u_x = j_x = v_x$.
V případě, že se body nacházejí na horizontální přímce, musíme odebrat delší hranu. To uděláme tak, že si body uspořádáme (a přejmenujeme) tak, že $a_x<b_x<c_x$, nebo $a_y, b_y, c_y$ (podle toho, jestli jsou na horizontální nebo vertikální přímce), kde $\{a, b, c\} = \{u, j, v\}$. Z grafu pak musíme odebrat hranu $\{a, c\}$.
\end{poznamka}


Pokud tyto body tvoří trojúhelník, pak tvoří cyklus délky tři. Bude-li tento cyklus bude mít váhu menší než ten, který jsme doposud našli, uložíme jej a vrátíme hranu $\{u, v\}$ do množiny hran $E$. Tento postup opakujeme, dokud nevyzkoušíme všechny hrany. Výsledkem bude trojúhelník s minimálním obvodem.
% Na \myref{obrázku}{obr:algoritmus_graf} můžete vidět, jak algoritmus funguje.


%    \begin{figure}[H]
%     \centering
%     \begin{tikzpicture}[node distance=2cm]
%         \node[draw, circle, fill, inner sep=1.5pt] (A) at (-6,1.4) {};
%         \node[left] at (A) {$a$};
%         \node[draw, circle, fill, inner sep=1.5pt] (B) at (5,1.5) {};
%         \node[below] at (B.south) {$b$};
%         \node[draw, circle, fill, inner sep=1.5pt] (C) at (-5,0) {};
%         \node[below] at (C.south) {$c$};
%         \node[draw, circle, fill, inner sep=1.5pt] (D) at (3,0) {};
%         \node[below] at (D.south) {$d$};
%         \node[draw, circle, fill, inner sep=1.5pt] (E) at (4,0.25) {};
%         \node[below] at (E.south) {$e$};
%         \node[draw, circle, fill, inner sep=1.5pt] (G) at (0,1.5) {};
%         \node[below] at (G.south) {$g$};
        
%         % Additional edges
%         \draw[lightgray, line width=0.5pt] (A) -- (E);
%         \draw[lightgray, line width=0.5pt] (A) -- (G);
%         \draw[lightgray, line width=0.5pt] (B) -- (C);
%         \draw[lightgray, line width=0.5pt] (B) -- (E);
%         \draw[lightgray, line width=0.5pt] (B) -- (G);
%         \draw[lightgray, line width=0.5pt] (C) -- (E);
%         \draw[lightgray, line width=0.5pt] (E) -- (G);
%         \draw[lightgray, line width=0.5pt] (A) -- (B);
%         \draw[lightgray, line width=0.5pt] (B) -- (D);
%         \draw[lightgray, line width=0.5pt] (A) -- (D);
%         \draw[lightgray, line width=0.5pt] (A) -- (C);
%         \draw[lightgray, line width=0.5pt] (D) -- (E);
%         \draw[mygreen, line width=1.5pt] (C) -- node[above] {2}(G);
%         \draw[mygreen, line width=1.5pt] (D) -- node[above] {1}(G);
%         \draw[myred, line width=1.5pt] (C) -- node[above] {1} (D); 
%         \node[draw, circle, fill, inner sep=1.5pt] (G) at (0,1.5) {};
%     \end{tikzpicture}
%     \caption{Příklad případu, kdy algoritmus vybral hranu $\{c, d\}$, odebral ji z množiny hran, a pomocí Dijkstrova algoritmu našel nejkratší cestu z bodu $c$ do bodu $d$.}
%     \label{obr:algoritmus_graf}
%     \end{figure}