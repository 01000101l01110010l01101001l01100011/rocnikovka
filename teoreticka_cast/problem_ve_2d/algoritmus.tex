\section{Algoritmus}
\label{sec:algoritmus}

Nechť $V$ je množina bodů v rovině a pro každé dva body $u, v \in V$ označme $d(u, v)$ jejich vzájemnou vzdálenost. Množinu hran označíme $E = \binom{V}{2}$ a váhu, neboli ohodnocení, nám určuje zobrazení $w$ dané předpisem $w(\{u, v\}) \coloneqq d(u, v)\mid \forall (u, v) \in E$. Nyní si můžeme definovat graf $G = (V, E, w)$. V prvním kroce zkontrolujeme, jestli nějaké tři body neleží na přímce. To provedeme tak, že pro každé body $a, b, c \in V$ zkontrolujeme, že: $(a_x \neq b_x \neq c_x) \lor (a_y \neq b_y \neq c_y) \lor ((a_x+k \neq b_x+k \neq c_x+k)\land(a_y+k \neq b_y+k \neq c_y+k))|k\in \mathbb{R}$. %Nah takhle ne, pres podobnosti trojuhelniku, nebo neco pres a_y=ka_x+b
Pokud nějaké tři body leží na přímce, musíme z grafu odebrat hranu mezi dvěma vnějšími body (pokud budeme mít přímku na které jsou body $a, b, c$, které splňují, že $a_x < b_x < c_x$, tak z grafu $G$ odebereme stranu $\{a, c\}$).
Tímto je příprava hotova. Následuje hledání cyklu délky 3 s minimální váhou.

Náhodně vybereme jednu hranu $\{u, v\} \in E$ a odebereme ji z množiny hran $E$. Dále potřebujeme najít cestu s minimální váhou mezi body $u$ a  $v$. K tomuto použijeme Dijkstrův algoritmus. Pokud cesta existuje, společně s hranou $\{u, v\}$ tvoří cyklus. Není těžké si rozmyslet, že tento cyklus bude mít délku tři. Cestu tvoří dvě hrany $\{u, j\}$ a $\{j, v\}$, které společně s hranou $\{u, v\}$ tvoří podgraf ve tvaru trojúhelníku $T = (V_T, E_T, w)$, kde $V_T = \{u, j, v\}$, $E_T = \{\{u,j\}, \{j,v\}, \{u,v\}\}$ a váha hran se zachová. Pokud trojúhelník (cyklus délky tři) bude mít celkovou váhu menší než ten, který jsme doposud našli, uložíme jej.
Vrátíme hranu $\{u, v\}$ do množiny hran $E$. Tento postup opakujeme dokud nevyzkoušíme všchny hrany. Výsledkem bude trojúhelník s minimálním obvodem.


   \begin{figure}[ht]
    \centering
    \begin{tikzpicture}[node distance=2cm]
        \node[draw, circle, fill, inner sep=1.5pt] (A) at (-6,3) {};
        \node[left] at (A) {$a$};
        \node[draw, circle, fill, inner sep=1.5pt] (B) at (-4,2.5) {};
        \node[below] at (B.south) {$b$};
        \node[draw, circle, fill, inner sep=1.5pt] (C) at (-5,0) {};
        \node[below] at (C.south) {$c$};
        \node[draw, circle, fill, inner sep=1.5pt] (D) at (3,0) {};
        \node[below] at (D.south) {$d$};
        \node[draw, circle, fill, inner sep=1.5pt] (E) at (6,3) {};
        \node[below] at (E.south) {$e$};
        \node[draw, circle, fill, inner sep=1.5pt] (G) at (0,2.5) {};
        \node[below] at (G.south) {$g$};
        
        % Additional edges
        \draw[lightgray, line width=0.5pt] (A) -- (E);
        \draw[lightgray, line width=0.5pt] (A) -- (G);
        \draw[lightgray, line width=0.5pt] (B) -- (C);
        \draw[lightgray, line width=0.5pt] (B) -- (E);
        \draw[lightgray, line width=0.5pt] (B) -- (G);
        \draw[lightgray, line width=0.5pt] (C) -- (E);
        \draw[lightgray, line width=0.5pt] (E) -- (G);
        \draw[lightgray, line width=0.5pt] (A) -- (B);
        \draw[lightgray, line width=0.5pt] (B) -- (D);
        \draw[lightgray, line width=0.5pt] (A) -- (D);
        \draw[lightgray, line width=0.5pt] (A) -- (C);
        \draw[lightgray, line width=0.5pt] (D) -- (E);
        \draw[mygreen, line width=1.5pt] (C) -- node[above] {2}(G);
        \draw[mygreen, line width=1.5pt] (D) -- node[above] {1}(G);
        \draw[myred, line width=1.5pt] (C) -- node[above] {1} (D); 

        
    \end{tikzpicture}
    \caption{Příklad toho kdy algoritmus vybral hranu $\{c, d\}$, odebral ji z množiny hran a pomocí Dijkstrova algoritmu našel nejkratší cestu z bodu $c$ do bodu $d$. (Pro čitelnost nejsou zobrazeny váhy ostatních hran. Předpokládejme, že jsou vyšší, než 3.)}
    \label{obr:algoritmus_graf}
\end{figure}

\subsection{Algoritmus v pseudokódu}

\begin{algorithm}
    \caption{Algoritmus na hledání cyklu délky 3.}
    \label{alg:algoritmus}

    \SetKwInOut{Input}{input}
    \SetKwInOut{Output}{output}
    \SetKw{KwReturn}{return}
   
    \Input{Množina bodů $V$ v rovině, kde každý bod je reprezentován jako dvojice souřadnic $(v_x, v_y)$}
    \Output{Trojúhelník $T = (V_T, E_T, w)$}
    \BlankLine
    \For{$u \in V$}{
        \For{$v \in V$}{
        $d(u, v) \leftarrow \sqrt{(v_x-u_x)^2+(v_y-u_y)^2}$\;
        $ w(\{u, v\}) \leftarrow d(u, v)$\;
        }
    }
    $ E \leftarrow{\binom{V}{2}}$\;
    $ G \leftarrow (V, E, w)$\;
    $min_T \leftarrow \infty$\;
    \For{$\{u, v\} \in E$}{
        $E \leftarrow E \setminus \{u, v\}$\;
        $j \leftarrow dijkstra(G, u, v)$\;
        \If{$d(u, j) + d(j, v) + d(u, v) < min_T$}{
            $min_T \leftarrow d(u, j) + d(j, v) + d(u, v)$\;
            $V_T \leftarrow \{u, j, v\}$\;
            $E_T \leftarrow \{\{u,j\}, \{j,v\}, \{u,v\}\}$\;
            $T \leftarrow (V_T, E_T, w)$\;
        }
        $E \leftarrow E \cup \{u, v\}$\;
    }
    \KwReturn{$T$}\;
   \end{algorithm}

