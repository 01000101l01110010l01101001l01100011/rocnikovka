Ve 2D budu hledat trojúhelník s minimálním obvodem. Nejprve mě napadl algoritmus, který úlohu převede na grafovou úlohu. Funguje tak, že postupně vybere každou hranu, odebere ji z grafu a z jednoho konce hrany spustí Dijkstrův algoritmus \autocite{benlFormalCorrectnessProofs1999}, který najde nejkratší cestu do druhého konce hrany (například algoritmus odebere hranu $\{u, v\}$ a algoritmus spustí Dijkstrův algoritmus z bodu $u$ do bodu $v$). Z \hyperref[tvrzeni:trojuhelnikova_nerovnost]{trojúhelníkové nerovnosti} víme, že taková cesta povede právě přes jeden bod. Trojúhelníková nerovnost v~našem grafu platí, protože je převzat z roviny a ohodnocení hran jsou vzdálenosti mezi body. Kdyby náš graf nebyl převzatý z~roviny, mohla by nastat situace na \myref{obrázku}{obr:troj_ner_graf}.

Pak jsem si uvědomil, že tento algoritmus je stejný jako ten, který vám představím v \myref{sekci}{sec:algoritmus}. Dijkstrův algoritmus tím pádem nepoužiji, protože na tento problém je zbytečně komplikovaný. Pro zájemce je Dijkstrův algoritmus popsán v \myref{příloze}{appendix:dijkstra}. Pro nový algoritmus je třeba dokázat \myref{tvrzení}{tvrzeni:podobnost_trojuhelniku} o podobnosti trojúhelníků, díky kterému zjistíme, zda body neleží na přímce.

\begin{figure}[H]
  \centering
  \begin{tikzpicture}[node distance=2cm]
    \node[draw, circle, fill, inner sep=1.5pt] (A) at (0,0) {};
    \node[below] at (A.south) {$u$};
    \node[draw, circle, fill, inner sep=1.5pt] (B) at (2,1.5) {};
    \node[below] at (B.south) {$j$};
    \node[draw, circle, fill, inner sep=1.5pt] (D) at (6,1.5) {};
    \node[below] at (D.south) {$v$};
    \node[draw, circle, fill, inner sep=1.5pt] (C) at (-3,1.5) {};
    \node[below] at (C.south) {$k$};
    \node[draw, circle, fill, inner sep=1.5pt] (E) at (8,-2) {};
    \node[below] at (E.south) {$x$};
    \node[draw, circle, fill, inner sep=1.5pt] (F) at (3.5,-1) {};
    \node[below] at (F.south) {$y$};
    \node[draw, circle, fill, inner sep=1.5pt] (G) at (-1,-1) {};
    \node[below] at (G.south) {$z$};

    \draw[myblue, line width=1.5pt] (A) -- node[above] {2} (B);
    \draw[myblue, line width=1.5pt] (B) -- node[above] {3} (D);
    \draw[myred, line width=1.5pt] (A) -- node[below] {7} (D);
    \draw[mygreen, line width=1.5pt] (A) -- node[above] {3} (C);
    \draw[mygreen, line width=1.5pt] (D) -- node[above, xshift=8pt] {17} (E);
    \draw[mygreen, line width=1.5pt] (F) -- node[above] {10} (E);
    \draw[mygreen, line width=1.5pt] (F) -- node[above] {9} (G);
    \draw[mygreen, line width=1.5pt] (C) -- node[below, xshift={0-5pt}] {15} (G);
    \draw[mygreen, line width=1.5pt] (D) -- node[above, xshift={0-5pt}] {4} (F);
    \draw[mygreen, line width=1.5pt] (A) -- node[above, xshift={0-5pt}] {1} (G);

  \end{tikzpicture}
  \caption{V obecném grafu nemusí platit \hyperref[tvrzeni:trojuhelnikova_nerovnost]{trojúhelníková nerovnost}.}
  \label{obr:troj_ner_graf}
\end{figure}