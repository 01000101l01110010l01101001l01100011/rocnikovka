\chapter{Problém v 1D}
\label{chap:problem_1D}
Problém v 1D je velice jednoduchý. Vstupem bude množina bodů $V \subset \mathbb{R}$ ve které budeme hledat nejkratší úsečku. Výstupem bude úsečka $\{a, b\} \mid a, b \in V$. Následujícím algoritmem můžeme tento problém vyřešit.

\section{Algoritmus}
\label{sec:algoritmus_1D}
\begin{enumerate}
    \item Nejprve body seřadíme.
    \item Projdeme všechny body $x_1\ldots x_n \in V$ a spočítáme vzdálenost všch po sobě jdoucích bodů: $|x_i-x_{i+1}|$, kde $i$ značí kolikátý bod procházíme. Tuto vzdálenost si uložíme, a pokud je menší, než ta doposud uložená, změníme ji. 
    \item Po tom, co projdeme všechny body, uložená minimální vzdálenost je řešením problému. 
\end{enumerate}
\begin{proof}
    Zřejmý. Algoritmus skončí, protože prochází konečnou množinu bodů a je korektní, protože spočítá všechny vzdálenosti po sobě jdoucích bodů a vybere tu minimální. 
\end{proof}
\begin{algorithm}
    \caption{Algoritmus na hledání úsečky s minimální délkou.}
    \label{alg:algoritmus_1D}

    \SetKwInOut{Input}{input}
    \SetKwInOut{Output}{output}
    \SetKw{KwReturn}{return}
   
    \Input{Množina bodů $V$ v 1D.}
    \Output{Dvojice bodů a jejich vzájemná vzdálenost.}
    \BlankLine
    \textcolor{red}{Mám tam psát sorting algoritmus nebo můžu počítat, že vstup už bude seřazený?} \;
    $d \leftarrow (x_0, x_\infty, \infty)$ \;
    \For{$x, i \in V$}{
        $ distance \leftarrow |x_i-x_{i+1}|$ \;
        \If{$distance < d$}{
            $d \leftarrow (x_i, x_{i+1}, distance)$ \;
        } 
    }
\end{algorithm}