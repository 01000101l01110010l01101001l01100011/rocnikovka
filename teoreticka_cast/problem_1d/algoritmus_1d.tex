\section{Algoritmus}
\label{sec:algoritmus_1D}
\begin{enumerate}
    \item Nejprve body seřadíme pomocí algoritmu quicksort.
    \item Projdeme všechny body $x_1\ldots x_n \in V$ a spočítáme vzdálenost všech po sobě jdoucích bodů: $d(x_i, x_{i+1})$, kde $i \in \{1,...,\#V-1\}$. Tuto vzdálenost si uložíme, a pokud je menší, než ta doposud uložená, změníme ji.
    \item Po tom, co projdeme všechny body, je uložená minimální vzdálenost řešením problému.
\end{enumerate}
\begin{tvrzeni}
    Algoritmus na hledání nejkratší úsečky je korektní.
\end{tvrzeni}
\begin{proof}
    Algoritmus skončí, protože prochází konečnou množinu bodů a je korektní, protože spočítá všechny vzdálenosti po sobě jdoucích bodů a vybere tu minimální.
\end{proof}


% \begin{center}
% \begin{figure}[H]
%     \label{obr:1d_obrazek}
%     \centering
%     \begin{tikzpicture}
%         \draw[->, line width = 1.5pt] (0,0) -- (10,0); % Line segment
%         \draw[color=myred, line width=2pt] (3.3, 0) -- (3.9, 0);
%         \foreach \x/\label in {1/$x_1$, 2/$x_2$, 3.3/$\textcolor{myred}{x_3}$, 3.9/$\textcolor{myred}{x_4}$, 5/$x_5$, 7/$x_6$, 9/$x_7$} % Numbers and labels
%             \draw[line width = 1.5pt] (\x,0.15) -- (\x,-0.15) node[below] {\label};
%     \end{tikzpicture}
%     \caption{Příklad problému v 1D}
% \end{figure}
% \end{center}
\begin{poznamka}
    Pseudokód je popis jednotlivých kroků v algoritmu s použitím základní logiky programovacích jazyků. Následuje náš algoritmus napsaný v pseudokódu.
\end{poznamka}

\begin{algorithm}[H]
    \caption{Algoritmus na hledání úsečky s minimální délkou.}
    \label{alg:algoritmus_1D}

    \SetKwInOut{Input}{input}
    \SetKwInOut{Output}{output}
    \SetKw{KwReturn}{return}

    \Input{množina čísel $V \subset \R$.}
    \Output{$a, b \in V, d(a, b)$}
    \BlankLine
    $points \leftarrow quicksort(V) $\;
    $shortest\_distance \leftarrow \infty$ \;
    $ closest\_points \leftarrow \emptyset $ \;
    \For{$i\in \{1, \ldots, \#V-1\}$}{
        \If{$d(x_i, x_{i+1}) < shortest\_distance$}{
            $shortest\_distance \leftarrow d(x_i, x_{i+1})$ \;
            $ closest\_points \leftarrow (x_i, x_{i+1})$ \;
        }
    }
    \Return{$closest\_points, shortest\_distance$}
\end{algorithm}