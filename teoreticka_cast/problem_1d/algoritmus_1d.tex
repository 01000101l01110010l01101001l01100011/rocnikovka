\section{Algoritmus}
\label{sec:algoritmus_1D}
\begin{enumerate}
    \item Nejprve body seřadíme.
    \item Projdeme všechny body $x_1\ldots x_n \in V$ a spočítáme vzdálenost všech po sobě jdoucích bodů: $d(x_i, x_{i+1})$, kde $i \in \{1,...,\#V-1\}$. Tuto vzdálenost si uložíme, a pokud je menší, než ta doposud uložená, změníme ji.
    \item Po tom, co projdeme všechny body, je uložená minimální vzdálenost řešením problému.
\end{enumerate}
\begin{tvrzeni}
    Algoritmus na hledání nejkratší úsečky je korektní.
\end{tvrzeni}
\begin{proof}
    Algoritmus skončí, protože prochází konečnou množinu bodů a je korektní, protože spočítá všechny vzdálenosti po sobě jdoucích bodů a vybere tu minimální.
\end{proof}
\begin{poznamka}
    Pseudokód je popis jednotlivých kroků v algoritmu s použitím základní logiky programovacích jazyků. Následuje náš algoritmus napsaný v pseudokódu.
\end{poznamka}
\begin{algorithm}
    \caption{Algoritmus na hledání úsečky s minimální délkou.}
    \label{alg:algoritmus_1D}

    \SetKwInOut{Input}{input}
    \SetKwInOut{Output}{output}
    \SetKw{KwReturn}{return}

    \Input{množina čísel $V \subset \mathbb{R}$.}
    \Output{dvojice bodů a jejich vzájemná vzdálenost.}
    \BlankLine
    points = \textcolor{red}{Seřazené body nějakým sorting algoritmem. set.pop() vytahuje nejmenší item?} \;
    $shortest \leftarrow \infty$ \;
    $ closest \leftarrow \emptyset $ \;
    \For{$i\in \{1, \ldots, \#V-1\}$}{
        \If{$d(x_i, x_{i+1}) < shortest$}{
            $shortest \leftarrow d(x_i, x_{i+1})$ \;
            $ closest \leftarrow (x_i, x_{i+1})$ \;
        }
    }
    \Return{closest, shortest}
\end{algorithm}