\chapter{Zobecnění na n dimenzí}
\label{chap:n_dimenzi}

Problém v $n$ dimenzích bude komplikovanější a časově náročnější. Problém si znovu převedeme na grafovou úlohu, kde $v \in V = \{(a_1, a_n)\}$ a $E = \binom{V}{2}$ nám tvoří ohodnocený graf $G = (V, E, w)$. V tomto grafu budeme hledat cyklus délky $n+1$ s minimální váhou. To uděláme tak, že v grafu nalezneme všechny cykly délky $n+1$ a budeme si je pamatovat seřazené podle jejich váhy. Pak vybereme cyklus s minimální váhou a ověříme, zda body opravdu tvoří polytop maximální dimenze (ve 2D jsme ověřovali, jestli body neleží na přímce). Polytop maximální dimenze se vyznačuje tím, že jeho body neleží ve stejné nadrovině. 


% Jinými slovy, potřebujeme aby všechny body tvořící cyklus byly:
% \begin{itemize}
%     \item ve stejné nejvyšší nadrovině
% \end{itemize}
% potřebujeme, aby všechny body ležely ve stejné nejvyšší nadrovině, neboli aby všechny body neležely ve stejné nadrovině menší, než maximální.



\section{Algoritmus v \texorpdfstring{$n$D}{nD}}
V této sekci Vám představím algoritmus, který řeší problém v $n$ dimenzích. Součástí algoritmu není algoritmus na počítání determinantu matice, protože existují různé algoritmy, a protože v \myref{sekci}{sec:program_nD}, kde jsem algoritmus naprogramoval, používám knihovnu \href{https://numpy.org/doc/stable/index.html}{NumPy}, která je optimalizovaná na matematické výpočty. 

Algoritmus dostane množinu bodů $V$. Dimenzi $n$ algoritmus zjistí podle počtu souřadnic jednoho bodu. Komentář v algoritmu označím jako // komentář.

\begin{algorithm}[H]
    \caption{Algoritmus na hledání polytopu maximální dimenze s minimálním obvodem.}
    \label{alg:algoritmus_nd}

    \SetKwInOut{Input}{input}
    \SetKwInOut{Output}{output}
    \SetKw{KwReturn}{return}
    \SetKwFunction{p}{$\rho$}
    \SetKwProg{Fn}{Function}{:}{}

    \Input{$V = \{x_1,\ldots, x_k\} \in \R$.}
    \Output{body, které tvoří polytop maximální dimenze s minimálním obvodem.}
    \BlankLine
    \Fn{\p{$\mathcal{P}$}}{
        $perimeter \leftarrow \sum_{i=1}^{n} \sum_{j=i+1}^{n+1}d(x_i, x_{j})$\;
        \KwReturn{$perimeter$}\;
    }
    \BlankLine

    $n \leftarrow \#x \in V$ \;
    $S \leftarrow \binom{V}{n+1}$\;
    $C \leftarrow []$\; 
    \For{$\mathcal{P} \in S$}{
      $C.append(\mathcal{P}, \rho(\mathcal{P}))$ \;
    }
    $C.sort()$  // Od nejmenšího po největší podle $\rho(\mathcal{P})$ \;
    $min_\mathcal{P} \leftarrow \emptyset$\;
    \While{$C$}{ %Od cyklu, který má minimální obvod. 
    $\mathcal{P}, \rho(\mathcal{P}) \leftarrow C[0]$ \;
        $x_1, \dots x_{n+1} \leftarrow \mathcal{P}$\;
        $A \leftarrow \begin{pmatrix}
          \x{1}{2} - \x{1}{1} & \x{2}{2} - \x{2}{1} & \cdots & \x{n}{2} - \x{n}{1} \\
          \x{1}{3} - \x{1}{1} & \x{2}{3} - \x{2}{1} & \cdots & \x{n}{3} - \x{n}{1} \\
          \vdots        & \vdots        & \ddots & \vdots        \\
          \x{1}{n_+_1} - \x{1}{1} & \x{2}{n_+_1} - \x{2}{1} & \cdots & \x{n}{n_+_1} - \x{n}{1} \\
        \end{pmatrix}$\;
        \If{$det(A)~!= 0$}{
          $min_\mathcal{P} \leftarrow \mathcal{P}$\;
          \KwReturn{$min_{\mathcal{P}}, \rho(\mathcal{P})$}\;
        }
        \Else{
          $C.remove(C[0])$\;
        }
    }
    \KwReturn{$\emptyset$}\;
   \end{algorithm}