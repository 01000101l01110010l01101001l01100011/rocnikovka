\chapter{Zobecnění na n dimenzí}
\label{chap:n_dimenzi}

Problém v $n$ dimenzích bude komplikovanější a časově náročnější. Problém si znovu převedeme na grafovou úlohu, kde $v \in V = \{(a_1, a_n)\}$ a $E = \binom{V}{2}$ nám tvoří ohodnocený graf $G = (V, E, w)$. V tomto grafu budeme hledat cyklus délky $n+1$ s minimální váhou. To uděláme tak, že v grafu nalezneme všechny cykly délky $n+1$ a budeme si je pamatovat seřazené podle jejich váhy. Pak vybereme cyklus s minimální váhou a ověříme, zda body opravdu tvoří polytop maximální dimenze (ve 2D jsme ověřovali, jestli body neleží na přímce). Polytop maximální dimenze se vyznačuje tím, že jeho body neleží ve stejné nadrovině. 


% Jinými slovy, potřebujeme aby všechny body tvořící cyklus byly:
% \begin{itemize}
%     \item ve stejné nejvyšší nadrovině
% \end{itemize}
% potřebujeme, aby všechny body ležely ve stejné nejvyšší nadrovině, neboli aby všechny body neležely ve stejné nadrovině menší, než maximální.



\section{Algoritmus}
V této sekci Vám představím algoritmus, který řeší problém v $n$ dimenzích. Součástí algoritmu není algoritmus na počítání determinantu matice, protože existují různé algoritmy, a protože v \myref{sekci}{sec:program_nD}, kde jsem algoritmus naprogramoval, používám knihovnu \href{https://numpy.org/doc/stable/index.html}{NumPy}, která je optimalizovaná na matematické výpočty. 


\begin{algorithm}[H]
    \caption{Algoritmus na hledání polytopu maximální dimenze s minimálním obvodem.}
    \label{alg:algoritmus_nd}

    \SetKwInOut{Input}{input}
    \SetKwInOut{Output}{output}
    \SetKw{KwReturn}{return}

    \Input{$V = \{x_1,\ldots, x_k\} \in \R$.}
    \Output{body, které tvoří polytop maximální dimenze s minimálním obvodem.}
    \BlankLine
    $C \leftarrow \binom{V}{n+1}$\;
    $C \leftarrow C.sort()$\; % TODO tohle nějak lépe..
    $min_c \leftarrow \emptyset$\;
    \For{$c\in C$}{ %Od cyklu, který má minimální obvod. 
      \If{$min_C > w(c)$}{
        $x_1, \dots x_{n+1} \leftarrow c$\;
        $A \leftarrow \begin{pmatrix}
          \x{1}{2} - \x{1}{1} & \x{2}{2} - \x{2}{1} & \cdots & \x{n}{2} - \x{n}{1} \\
          \x{1}{3} - \x{1}{1} & \x{2}{3} - \x{2}{1} & \cdots & \x{n}{3} - \x{n}{1} \\
          \vdots        & \vdots        & \ddots & \vdots        \\
          \x{1}{n_+_1} - \x{1}{1} & \x{2}{n_+_1} - \x{2}{1} & \cdots & \x{n}{n_+_1} - \x{n}{1} \\
        \end{pmatrix}$\;
        \If{$det(A) != 0$}{
          $min_c \leftarrow c$\;
        }
      }
    }
    \KwReturn{$min_c$, $w(min_c)$}\;
   \end{algorithm}