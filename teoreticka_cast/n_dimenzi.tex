\chapter{Zobecnění na \texorpdfstring{$n$}{n} dimenzí}
\label{chap:n_dimenzi}
\section{Pomocné definice}
Tyto definice nám umožní vyřešit problém zobecněný na $n$ dimenzí.

\begin{definition}[Nadrovina]
  \label{definice:nadrovina}
  Nadrovina je podprostor $\R^n$ dimenze $n-1$.
\end{definition}

\begin{definition}[Lineární kombinace]
  \label{definice:linearni-kombinace}
  Bod $a_1x_1+a_2x_2+\dots+a_kx_k$, kde $a_1, \dots, a_k \in \R$ je lineární kombinací bodů $x_1, \dots, x_k \in \R^n$. \autocite[67]{jindrichbecvarLinearniAlgebra2005}
\end{definition}

\begin{definition}[Lineární závislost]
  \label{definice:linearni-zavislost}
  Množina bodů $x_1, \dots, x_n \in \mathbb{R}$ se nazývá lineárně závislá, jestliže lze nějaký bod z této množiny vyjádřit jako lineární kombinaci ostatních bodů této množiny. \autocite[78]{jindrichbecvarLinearniAlgebra2005}
\end{definition}

\begin{definition}[Afinní závislost]
  \label{definice:afinni-zavislost}
  Body $x_1, \ldots, x_k \in \R^n$ jsou afinně závislé právě tehdy, když jejich rozdíly $x_2-x_1, x_3-x_1, \dots, x_k-x_1$ jsou lineárně závislé. Geometricky to znamená, že body leží v jedné nadrovině.  
\end{definition}

\begin{definition}[Determinant matice]
  \label{definice:determinant}
  Determinant je zobrazení značené $det$, které posílá čtvercové matice na reálné číslo. Determinant jde interpretovat geometricky; lze chápat jako objem obecného $n$-rozměrného rovnoběžnostěnu. Důležitá vlastnost determinantu je, že řádky nebo sloupce matice $A$ jsou lineárně závislé právě tehdy, když $det(A)=0$. \autocite[224]{danmargalitInteractiveLinearAlgebra2019}

  Pro hlubší pochopení determinantů vizte \autocite[164]{jindrichbecvarLinearniAlgebra2005} nebo \autocite[187]{danmargalitInteractiveLinearAlgebra2019}.
\end{definition}

% \begin{definition}[Objem obecného $n$-rozměrného rovnoběžnostěnu]
%   \label{definice:objem-rovnobeznostenu}
%   Označme $\mathcal{R}_n$ ja\-ko $n$-dimenzionální rovnoběžnostěn, který je určen pomocí bodů $x^1, \ldots, x^k \in \R^n$. 
%   Uvědomte si, že pokud jsou vektory $v_1, \dots, v_n$ lineárně závislé, rovnoběžnostěn leží v podprostoru dimenze $n-k$, kde $k \in [1, n-1]$.
% \end{definition}

% Definovat afinní závislost, geometricky znamená, že body leží v jedné nadrovině
% Definovat afinní kombinaci i s lineární kombinací

% Ocitovat, že body jsou afinně závislé právě tehdy, když jsou rozdily bodu linearne zavisle
% Jak poznat
% Determinant je objem rovnoběžnostěnu? nějak ale kvalitně napsat a ocitovat to. 


\section{Řešení problému v \texorpdfstring{$n$D}{nD}}
\label{sec:reseni_1D}
Problém v $n$ dimenzích bude komplikovanější a časově náročnější. Úlohu znovu převedeme na grafovou úlohu, kde $V = \{x_1,\ldots, x_k\} \in \R^n$ a $E = \binom{V}{2}$ nám tvoří ohodnocený graf $G = (V, E, w)$. Cílem je získat cyklus délky $n+1$ s minimální váhou.

Toho docílíme tak, že v grafu nalezneme všechny cykly délky $n+1$ a budeme si je pamatovat seřazené podle jejich váhy. Pak vybereme cyklus s minimální váhou a ověříme, že jeho vrcholy opravdu tvoří polytop maximální dimenze (jako ve 2D, kde jsme ověřovali, jestli body neleží na přímce).

Polytop maximální dimenze se vyznačuje tím, že jeho body neleží ve stejné \hyperref[definice:nadrovina]{nadrovině}. Z \myref{definice}{definice:afinni-zavislost} víme, že pokud rozdíly bodů jsou lineárně závislé, tak body v jedné nadrovině leží. Z toho vyplývá, že chceme, aby body nebyly afinně závislé, neboli aby tvořily polytop maximální dimenze.  

\begin{poznamka} 
  \label{poznamka:varovani_index}
  Značení $\x{n}{k}$ znamená $n$-tá souřadnice $k$-tého bodu, například $\x{2}{3}$ je druhá souřadnice třetího bodu. Tím pádem $\x{n}{k} - \x{n}{1}$ je rozdíl $n$-té souřadnice bodu $x_k$ a $x_1$. 
\end{poznamka}

Abychom zjistili, zda jsou body afinně závislé, musíme spočítat jejich rozdíly a zjistit, jestli jsou lineárně závislé. Od bodů $x_2, \ldots, x_k \in \R^n$, kde  $k, n \in \mathbb{N}$ (Číslo $k$ v tomto případě značí počet bodů a $n$ je mocnina značící dimenzi. Vizte \myref{poznámku}{poznamka:varovani_index}.) odečteme první bod $x_1$. Tím nám vzniknou rozdíly $x_2-x_1, x_3-x_1, \dots, x_k-x_1$, které umístíme do matice $A$. 

% \x{vlevo}{vpravo} vlevo = souradnice, vpravo = kolikaty vektor

\begingroup
\renewcommand*{\arraystretch}{1.25}
% \setlength\arraycolsep{0pt}
\delimitershortfall=0pt
\begin{equation*}
  A =
  \begin{pmatrix}
    x_2-x_1 \\
    x_3-x_1 \\
    \vdots  \\
    x_k-x_1
  \end{pmatrix}
  =
  \begin{pmatrix}
    \x{1}{2} - \x{1}{1} & \x{2}{2} - \x{2}{1} & \cdots & \x{n}{2} - \x{n}{1} \\
    \x{1}{3} - \x{1}{1} & \x{2}{3} - \x{2}{1} & \cdots & \x{n}{3} - \x{n}{1} \\
    \vdots        & \vdots        & \ddots & \vdots        \\
    \x{1}{k} - \x{1}{1} & \x{2}{k} - \x{2}{1} & \cdots & \x{n}{k} - \x{n}{1} \\
  \end{pmatrix}
\end{equation*}
\endgroup
Teď, když máme matici, můžeme vypočítat $det(A)$. Podle \myref{definice}{definice:determinant}, pokud vyjde determinant 0, znamená to, že pak rozdíly bodů jsou lineárně závislé. Z \myref{definice}{definice:afinni-zavislost} ale víme, že pokud tyto rozdíly jsou lineárně závislé, pak body jsou afinně závislé, což znamená, že leží ve stejné nadrovině.  

Ještě napíšu geometrické odvození..


\section{Algoritmus v \texorpdfstring{$n$D}{nD}}
V této sekci Vám představím algoritmus, který řeší problém v $n$ dimenzích. Součástí algoritmu není algoritmus na počítání determinantu matice, protože existují různé algoritmy, a protože v \myref{sekci}{sec:program_nD}, kde jsem algoritmus naprogramoval, používám knihovnu \href{https://numpy.org/doc/stable/index.html}{NumPy}, která je optimalizovaná na matematické výpočty. 

Algoritmus dostane množinu bodů $V$. Dimenzi $n$ algoritmus zjistí podle počtu souřadnic jednoho bodu. Komentář v algoritmu označím jako // komentář.

\begin{algorithm}[H]
    \caption{Algoritmus na hledání polytopu maximální dimenze s minimálním obvodem.}
    \label{alg:algoritmus_nd}

    \SetKwInOut{Input}{input}
    \SetKwInOut{Output}{output}
    \SetKw{KwReturn}{return}
    \SetKwFunction{p}{$\rho$}
    \SetKwProg{Fn}{Function}{:}{}

    \Input{$V = \{x_1,\ldots, x_k\} \in \R$.}
    \Output{body, které tvoří polytop maximální dimenze s minimálním obvodem.}
    \BlankLine
    \Fn{\p{$\mathcal{P}$}}{
        $perimeter \leftarrow \sum_{i=1}^{n} \sum_{j=i+1}^{n+1}d(x_i, x_{j})$\;
        \KwReturn{$perimeter$}\;
    }
    \BlankLine

    $n \leftarrow \#x \in V$ \;
    $S \leftarrow \binom{V}{n+1}$\;
    $C \leftarrow []$\; 
    \For{$\mathcal{P} \in S$}{
      $C.append(\mathcal{P}, \rho(\mathcal{P}))$ \;
    }
    $C.sort()$  // Od nejmenšího po největší podle $\rho(\mathcal{P})$ \;
    $min_\mathcal{P} \leftarrow \emptyset$\;
    \While{$C$}{ %Od cyklu, který má minimální obvod. 
    $\mathcal{P}, \rho(\mathcal{P}) \leftarrow C[0]$ \;
        $x_1, \dots x_{n+1} \leftarrow \mathcal{P}$\;
        $A \leftarrow \begin{pmatrix}
          \x{1}{2} - \x{1}{1} & \x{2}{2} - \x{2}{1} & \cdots & \x{n}{2} - \x{n}{1} \\
          \x{1}{3} - \x{1}{1} & \x{2}{3} - \x{2}{1} & \cdots & \x{n}{3} - \x{n}{1} \\
          \vdots        & \vdots        & \ddots & \vdots        \\
          \x{1}{n_+_1} - \x{1}{1} & \x{2}{n_+_1} - \x{2}{1} & \cdots & \x{n}{n_+_1} - \x{n}{1} \\
        \end{pmatrix}$\;
        \If{$det(A)~!= 0$}{
          $min_\mathcal{P} \leftarrow \mathcal{P}$\;
          \KwReturn{$min_{\mathcal{P}}, \rho(\mathcal{P})$}\;
        }
        \Else{
          $C.remove(C[0])$\;
        }
    }
    \KwReturn{$\emptyset$}\;
   \end{algorithm}