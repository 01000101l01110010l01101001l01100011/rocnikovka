\section{Pomocné definice a tvrzení}
\label{sec:pomocne-definice}
Tyto definice nám umožní vyřešit problém zobecněný na $n$ dimenzí.

\begin{definition}[Nadrovina]
  \label{definice:nadrovina}
  Nadrovina je podprostor $\R^n$ dimenze $n-1$.
\end{definition}

\begin{definition}[Lineární kombinace]
  \label{definice:linearni-kombinace}
  Bod $a_1x_1+\dots+a_kx_k$, kde $a_1, \dots, a_k \in \R$ je lineární kombinací bodů $x_1, \dots, x_k \in \R^n$. \autocite[67]{jindrichbecvarLinearniAlgebra2005}
\end{definition}

\begin{definition}[Lineární závislost]
  \label{definice:linearni-zavislost}
  Množina bodů $x_1, \dots, x_n \in \mathbb{R}$ se nazývá lineárně závislá, jestliže lze nějaký bod z této množiny vyjádřit jako lineární kombinaci ostatních bodů této množiny. \autocite[78]{jindrichbecvarLinearniAlgebra2005}
\end{definition}

\begin{definition}[Afinní závislost]
  \label{definice:afinni-zavislost}
  Body $x_1, \ldots, x_k \in \R^n$ jsou afinně závislé právě tehdy, když jejich rozdíly $x_2-x_1, x_3-x_1, \dots, x_k-x_1$ jsou lineárně závislé. Geometricky to znamená, že body leží v jedné nadrovině. \autocite[4]{matousekConvexity2003}
\end{definition}

\begin{tvrzeni}[Determinant matice]
  \label{tvrzeni:determinant}
  Determinant je zobrazení značené $det$, které posílá čtvercové matice na reálné číslo. Determinant jde interpretovat geometricky; lze chápat jako objem $n$-rozměrného rovnoběžnostěnu. Důležitá vlastnost determinantu je, že řádky nebo sloupce matice $A$ jsou lineárně závislé právě tehdy, když $det(A)=0$. \autocites[224]{danmargalitInteractiveLinearAlgebra2019}[4]{matousekConvexity2003}
  Pro hlubší pochopení determinantů vizte \autocite[164]{jindrichbecvarLinearniAlgebra2005} nebo \autocite[187]{danmargalitInteractiveLinearAlgebra2019}.
\end{tvrzeni}

% \begin{definition}[Objem obecného $n$-rozměrného rovnoběžnostěnu]
%   \label{definice:objem-rovnobeznostenu}
%   Označme $\mathcal{R}_n$ ja\-ko $n$-dimenzionální rovnoběžnostěn, který je určen pomocí bodů $x^1, \ldots, x^k \in \R^n$. 
%   Uvědomte si, že pokud jsou vektory $v_1, \dots, v_n$ lineárně závislé, rovnoběžnostěn leží v podprostoru dimenze $n-k$, kde $k \in [1, n-1]$.
% \end{definition}