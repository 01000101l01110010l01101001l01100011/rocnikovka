\section{Řešení problému v \texorpdfstring{$n$D}{nD}}
\label{sec:reseni_1D}

Problém v $n$ dimenzích bude komplikovanější a časově náročnější. Z množiny bodů $V = \{x_1,\ldots, x_k\} \in \R^n$ vytvoříme množinu všech cyklů délky $n+1$: $C = \binom{V}{n+1}$. Cykly $\{c_1, \dots, c_i\} \in C$, kde $i=\#C$, očíslujeme podle jejich váhy ($c_1$ má minimální váhu). Poté vybereme cyklus s minimální váhou a ověříme, že jeho vrcholy opravdu tvoří polytop maximální dimenze (jako ve 2D, kde jsme ověřovali, jestli body neleží na přímce).

Polytop maximální dimenze se vyznačuje tím, že jeho body neleží ve stejné \hyperref[definice:nadrovina]{nadrovině}. Z \myref{definice}{definice:afinni-zavislost} víme, že pokud rozdíly bodů jsou lineárně závislé, tak body v jedné nadrovině leží. Z toho vyplývá, že chceme, aby body nebyly afinně závislé, neboli aby tvořily polytop maximální dimenze.  

\begin{poznamka} 
  \label{poznamka:varovani_index}
  Značení $\x{n}{k}$ znamená $n$-tá souřadnice $k$-tého bodu, například $\x{2}{3}$ je dru\-há souřadnice třetího bodu, tudíž $\x{n}{k} - \x{n}{1}$ je rozdíl $n$-té souřadnice bodu $x_k$ a $x_1$. 
\end{poznamka}

Abychom zjistili, zda jsou body afinně závislé, musíme spočítat jejich rozdíly a zjistit, jestli jsou lineárně závislé. Od bodů $x_2, \ldots, x_{n+1} \in \R^n$, kde  $n \in \mathbb{N}$ odečteme první bod $x_1$. Tím nám vzniknou rozdíly $x_2-x_1, x_3-x_1, \dots, x_{n+1}-x_1$, které umístíme do matice $A$. 

% \x{vlevo}{vpravo} vlevo = souradnice, vpravo = kolikaty vektor

\begingroup
\renewcommand*{\arraystretch}{1.25}
% \setlength\arraycolsep{0pt}
\delimitershortfall=0pt
\begin{equation*}
  A =
  \begin{pmatrix}
    x_2-x_1 \\
    x_3-x_1 \\
    \vdots  \\
    x_k-x_1
  \end{pmatrix}
  =
  \begin{pmatrix}
    \x{1}{2} - \x{1}{1} & \x{2}{2} - \x{2}{1} & \cdots & \x{n}{2} - \x{n}{1} \\
    \x{1}{3} - \x{1}{1} & \x{2}{3} - \x{2}{1} & \cdots & \x{n}{3} - \x{n}{1} \\
    \vdots        & \vdots        & \ddots & \vdots        \\
    \x{1}{n_+_1} - \x{1}{1} & \x{2}{n_+_1} - \x{2}{1} & \cdots & \x{n}{n_+_1} - \x{n}{1} \\
  \end{pmatrix}
\end{equation*}
\endgroup
Teď, když máme matici, vypočítáme $det(A)$. Podle \myref{definice}{definice:determinant}, pokud vyjde determinant $0$, znamená to, že pak rozdíly bodů jsou lineárně závislé. Z \myref{definice}{definice:afinni-zavislost} ale víme, že pokud tyto rozdíly jsou lineárně závislé, pak body jsou afinně závislé, což znamená, že leží ve stejné nadrovině.  

Nakonec zbývá spočítat obvod tohoto $n$-dimenzionálního polytopu. Tento polytop si označíme jako $\mathcal{P}_n = \{x_1, \dots, x_{n+1}\}$. Teď je otázkou kolik hran opravdu tvoří obvod. (Například: ve $3$D tělesová úhlopříčka krychle netvoří obvod.) Je jich přesně $\binom{n+1}{2}$. Toto číslo pochází ze vzorce $\binom{n+1}{m+1}$ \autocite[120]{coxeter1973regular},
který znamená počet $m$-dimenzionálních podpolytopů (dále $\mathcal{P}^{\prime}_m$) v $n$-rozměrném polytopu ($\mathcal{P}_n$) za předpokladu, že $\mathcal{P}_n$ má minimální počet bodů. To je $n+1$. 
Když víme že vnějších hran máme $\binom{n+1}{2}$, musíme zjistit které hrany z celkové množiny hran to jsou. Jsou to všechny, protože náš polytop má přesně $n+1$ vrcholů, proto můžeme místo $n+1$ dosadit množinu bodů polytopu ($P = \{x_1, \dots, x_{n+1}\}$) a tím dostaneme $\binom{P}{2}$, což jsou všechny dvouprvkové podmnožiny z P, neboli všechny úsečky polytopu.