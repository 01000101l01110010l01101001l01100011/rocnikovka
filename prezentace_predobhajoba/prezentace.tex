\documentclass{beamer}
\usetheme{CambridgeUS}
\usecolortheme{spruce}

\usepackage[utf8]{inputenc}
\usepackage[T1]{fontenc}
\usepackage[czech]{babel}
\usepackage{csquotes}
\usepackage{amsmath}
\usepackage{tikz}

\usetikzlibrary {perspective, shapes}
\usepackage{caption}
\usepackage{subcaption}

\definecolor{myred}{RGB}{170, 55, 55}
\definecolor{myblue}{RGB}{88,118,199}
\definecolor{mygreen}{RGB}{55, 120, 55}
\definecolor{graph}{RGB}{102,138,224}
\definecolor{graph2}{RGB}{224,102,138}

\definecolor{green1}{RGB}{5,147,84}
\definecolor{green2}{RGB}{2,123,54}
\definecolor{green3}{RGB}{29,105,68}

\definecolor{white1}{RGB}{220,250,250}
\definecolor{vyzkum_color}{RGB}{119,96,210}

\setbeamercolor{structure}{fg=mygreen} % Change the color of structural elements to "myred"
% \setbeamercolor{block title}{bg=blue!70!black, fg=white} % Change the color of block titles
\setbeamercolor{block title}{bg=myblue, fg=white}

\setbeamercolor{palette primary}{bg=green1}
\setbeamercolor{palette secondary}{bg=green2}
\setbeamercolor{palette tertiary}{bg=green3}


\setbeamercolor{title in head/foot}{fg=white1}
\setbeamercolor{author in head/foot}{fg=white1}
\setbeamercolor{date in head/foot}{fg=white1}
\setbeamercolor{title}{fg=white}
\setbeamercolor{section in head/foot}{fg=white1}
\setbeamercolor{subsection in head/foot}{fg=white1}

\newenvironment<>{vyzkum}[1]{%
  \setlength{\textwidth}{1\textwidth}
  \setbeamercolor{block title}{fg=white,bg=vyzkum_color}%
  \begin{block}#2{#1}}{\end{block}}



\usepackage{enumitem,amssymb}
\newlist{todolist}{itemize}{2}
\setlist[todolist]{label=$\square$}
\usepackage{pifont}
\newcommand{\cmark}{\ding{51}}%
\newcommand{\xmark}{\ding{55}}%
\newcommand{\done}{\rlap{$\square$}{\raisebox{2pt}{\large\hspace{1pt}\cmark}}%
\hspace{-2.5pt}}
\newcommand{\wontfix}{\rlap{$\square$}{\large\hspace{1pt}\xmark}}

\newtheorem{definice}{Definice}

\begin{document}

\title[Ročníková práce]{Hledání polytopu maximální dimenze a minimálním obvodem s vrcholy v dané množině bodů.}
\author{Eric Dusart}
\date{9. ledna 2024}

\begin{frame}
    \titlepage
\end{frame}

\begin{frame}
    \frametitle{Obsah}
    \tableofcontents
\end{frame}

\section{Co je to polytop a o čem píšu}
\subsection{Definice polytopu}
\begin{frame}
    \frametitle{Definice polytopu}
    \begin{definice}[Polytop]
        amogus
    \end{definice}

    \begin{itemize}
        \item Geometrický útvar.
        \item Zobecnění mnohoúhelník na n dimenzí.
        \item Bod, úsečka, mnohoúhelník, mnohostěn, polychor, \dots
    \end{itemize}

    \begin{figure}
        \centering
        \begin{subfigure}[b]{0.3\textwidth}
            \centering
            \begin{tikzpicture}
                \draw[myred, thick] (0,0) -- (2,0);
                \fill[myred] (0,0) circle (1pt);
                \fill[myred] (2,0) circle (1pt);
            \end{tikzpicture}
            \caption{Úsečka}
            \label{fig:line_segment}
        \end{subfigure}
        \begin{subfigure}[b]{0.3\textwidth}
            \centering
            \begin{tikzpicture}
                \draw[myblue, thick] (0,0) -- (1,1) -- (2,0) -- cycle;
                \fill[myblue] (0,0) circle (1pt);
                \fill[myblue] (1,1) circle (1pt);
                \fill[myblue] (2,0) circle (1pt);
            \end{tikzpicture}
            \caption{Mnohoúhelník}
            \label{fig:polygon}
        \end{subfigure}
        \begin{subfigure}[b]{0.3\textwidth}
            \centering
            \begin{tikzpicture}[3d view]
                \draw[green2, thick, dotted] (1,0,0) -- (0,1,0);
                \draw[green2, thick] (0,1,0) -- (0,0,2);
                \draw[green2, thick] (0,-1,0) -- (1,0,0);
                \draw[green2, thick] (1,0,0) -- (0,0,2);
                \draw[green2, thick] (0,-1,0) -- (0,0,2);
                \draw[green2, thick] (0,1,0) -- (0,-1,0);
                \fill[green3] (1,0,0) circle (1pt);
                \fill[green3] (0,1,0) circle (1pt);
                \fill[green3] (0,-1,0) circle (1pt);
                \fill[green3] (1,0,0) circle (1pt);
                \fill[green3] (0,0,1.99) circle (1pt);
            \end{tikzpicture}
            \caption{Mnohostěn}
            \label{fig:polygonds}
        \end{subfigure}
    \end{figure}

\end{frame}
\section{Moje práce}
\begin{frame}
    \frametitle{Moje práce}
    \begin{vyzkum}{Výzkumná otázka}
        Jak najít polytop maximální dimenze a minimálním obvodem s vrcholy v dané množině bodů?
    \end{vyzkum}
    \begin{description}
        \item[Vstup:] Množina bodů $v \in V$, $n \in \mathbb{N}$ značící dimenzi bodů: $v \in \mathbb{R}^n$.
        \item[Krok 1:] Převod na grafovou úlohu\footnote[1]{\tiny V grafu nezáleží na uspořádání bodů, ani na vzdálenostech.}.
        \item[Krok 2:] Spočítat všechny vzdálenosti mezi body.
        \item[Krok 3:] Pomocí algoritmu najít cyklus délky $n+1$ (polytop). 
        \item[Cíl:] Polytop maximální dimenze s minimálním obvodem. 
    \end{description}
    
\end{frame}
\subsection{Postup}
\begin{frame}
    \frametitle{Můj postup v psaní práce}
    \begin{todolist}
        \item Problém ve 2D
        \begin{todolist}
            \item[\done] Najít algoritmus.
            \item Dokázat, že funguje.
            \item Naprogramovat algoritmus.
        \end{todolist}
        \item Zobecnění na $n$ dimenzí
        \begin{todolist}
            \item Najít algoritmus.
            \item Dokázat, že funguje.
            \item Naprogramovat algoritmus.
        \end{todolist}
    \end{todolist}
\end{frame}

\section{Závěr}
\subsection{Děkuji za pozornost}
\begin{frame}
    \frametitle{To je prozatím všechno}
    \pgfdeclarelayer{background layer}
    \pgfdeclarelayer{foreground layer}
    \pgfsetlayers{background layer,main,foreground layer}

    \begin{figure}
        \centering
        \begin{tikzpicture}
            \begin{pgfonlayer}{foreground layer}
            \foreach \i in {1,...,11} {
                \pgfmathsetmacro{\x}{rand*6}
                \pgfmathsetmacro{\y}{rand*3}
                \node[fill=graph2,inner sep=2pt, circle](x\i) at (\x, \y){};
            }
            % \fill[red] (x3) circle (2pt);
            \end{pgfonlayer}
            \begin{pgfonlayer}{background layer}
                

            \foreach \i in {1,...,11} {
                \foreach \j in {1,...,11}{
                    \draw[graph] (x\i) -- (x\j);
                }
            }
            \end{pgfonlayer}
        \end{tikzpicture}
        \caption{Náhodný úplný graf $K_{11}$}
        \label{fig:body}
    \end{figure}
\end{frame}


\end{document}
