\chapter{Programování algoritmů}
\label{chap:programovani}

Moje praktická část bude programování problému, o kterým jsem psal v teoretické části. K programování použiji programovací jazyk \href{https://www.python.org/}{python}, ve kterém už umím programovat. Python nabízí mnoho různých knihoven, které dokážou velmi ušetřit práci. Knihovna je sbírka funkcí a metod, které jsou předem napsané a můžeme je získat jako doplněk pro snadnější vývoj softwaru. Ve své práci použiji knihovnu \href{https://networkx.org/}{networkx}, která je optimalizovaná pro práci s grafy. Použiji ji, protože nabízí lehké ukládání a získávání dat z grafu.

\section{Dijkstrův algortimus}
\label{sec:dijkstra_program}

Jelikož Dijkstrův algoritmus je součástí našeho algoritmu na řešení problému ve 2D, tak ho budu muset naprogramovat. Knihovna \href{https://networkx.org/}{networkx} má také k dispozici Dijkstrův algoritmus, ale pro náš problém je potřeba lehce modifikovaný; musí se zastavit v ten moment, kdy se vybere cílový bod za aktuální bod. 

\begin{lstlisting}[style=me]
def dijkstra_triangle(G:Graph, start, end):
    Q = set()
    distances = {} 
    middle_point = {} 
    for vertex in G:
        distances[vertex] = float('inf')
        middle_point[vertex] = None
        Q.add(vertex)

    distances[start] = 0
    while Q: 
        actual = min(Q, key=lambda vertex: distances[vertex])
        if actual == end:
            return distances[end], [start, middle_point[end], end]
        Q.remove(actual)
        neighbors = Q
        if actual == start:
            neighbors.remove(end)
            calculated_distance = distances[actual] + G.get_edge_data(actual, neighbor)['weight']
            if calculated_distance < distances[neighbor]:
                distances[neighbor] = calculated_distance
                middle_point[neighbor] = actual
    return distances[end], [start, middle_point[end], end]
\end{lstlisting}
\section{Program v \texorpdfstring{$1$D}{nD}}
\label{sec:program_1D}
\section{Program ve 2D}
\label{sec:program_2D}

Ve 2D je problém trochu těžší, protože hledáme tři body, které tvoří trojúhelník s~minimálním obvodem a musíme pouze kontrolovat, jestli nejsou na přímce. Tento program dostane jako vstup množinu bodů v $\R^2$ a výstupem budou tři body tvořící trojúhelník s minimálním obvodem. 
\begin{mdframed}[style=MyFrame]
\begin{lstlisting}[style=metoo]
from math import dist  # Importuje funkci dist z knihovny math pro výpočty.
from networkx import Graph, neighbors  # Importuje třídu Graph a funkci neighbors z~knihovny networkx pro práci s grafy.
from itertools import combinations
import numpy as np  # Import dalších užitečných knihoven.

def find_path(G:Graph, start, end):
  min_path = None, None, None, float("inf")  # Proměnná pro nejkratší cestu s nekonečnou vzdáleností.
  for x in neighbors(G, start) :  # Prochází sousedy počátečního bodu v grafu G.
    path = G.get_edge_data(start, x)["weight"] + G.get_edge_data(end, x)["weight"]  #~Spočítá~vzdálenost~cesty~přes~aktuálního~souseda.
    if float(min_path[3]) > path:  # Pokud je nová cesta kratší než dosavadní.
      min_path = start, x, end, path  # Aktualizuje nejkratší cestu.
  return min_path  # Vrátí nejkratší cestu.

def algorithm(V:np.array):
  G = Graph()  # Prázdný graf.
  subsets = combinations(V, 2) # Všechny dvouprvkové seznamy z~V.
  for edge in subsets: # Pro hranu v subsets.
    G.add_weighted_edges_from([((edge[0][0], edge[0][1]), (edge[1][0], edge[1][1]), dist(edge[0], edge[1]))])  # Přidá hranu edge do grafu G s její váhou.

  min_triangle = None  # Proměnná pro trojúhelník s minimální váhou.
  min_triangle_weight = float("inf")  # Proměnná pro minimální váhu trojúhelníku.
  for edge in G.edges():  
    edge_weight = G.get_edge_data(edge[0], edge[1])["weight"]        #~Váha~hrany~edge.
    G.remove_edge(edge[0], edge[1])  # Odebere hranu z grafu.
    u, j, v, weight= find_path(G, edge[0], edge[1])  # Najde nejkratší cestu mezi body edge[0] a edge[1].
    if (v[1]-u[1])*(j[0]-u[0]) == (j[1]-u[1])*(v[0]-u[0]):  # Pokud body u, j, v leží na jedné přímce.
      if dist(u, j) > dist(u, v):  # Pokud je vzdálenost mezi body u a j větší než vzdálenost mezi body u a v.
        G.remove_edge(u, j)  # Odebere hranu mezi body u a j.
      elif dist(j, v) > dist(u, v):  # Pokud je vzdálenost mezi body j a v větší než vzdálenost mezi body u a v.
        G.remove_edge(v, j)  # Odebere hranu mezi body v a j.
      else:
        continue  # Pokračuje na další iteraci cyklu.
    elif weight + edge_weight < min_triangle_weight:  # Pokud je váha cesty plus váha hrany menší než váha nejmenšího trojúhelníku.
      min_triangle = (u, j, v)  # Aktualizuje nejmenší trojúhelník.
      min_triangle_weight = weight + edge_weight  # Aktualizuje váhu nejmenšího trojúhelníku.
    G.add_weighted_edges_from([(u, v, dist(u, v))])  # Přidá hranu u, v zpátky do grafu s její váhou.

  return min_triangle, min_triangle_weight  # Vrátí trojúhelník a jeho obvod.
\end{lstlisting}
\end{mdframed}
\input{prakticka_cast/programovani/program_nD}