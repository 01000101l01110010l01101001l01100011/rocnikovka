\chapter{Programování algoritmů}
\label{chap:programovani}

Moje praktická část bude programování problému, o kterým jsem psal v \hyperref[part:teoreticka-cast]{teoretické části}. K programování použiji programovací jazyk \href{https://www.python.org/}{python}, ve kterém už umím programovat. Python nabízí mnoho různých knihoven, které dokážou velmi ušetřit práci. Knihovna je sbírka funkcí a metod, které jsou předem napsané a můžeme je získat jako doplněk pro snadnější vývoj softwaru. Například ve své práci použiji knihovnu \href{https://networkx.org/}{networkx}, která je optimalizovaná pro práci s grafy. Použiji ji, protože nabízí lehké ukládání a získávání dat z grafu.

Cílem programů je, aby fungovaly a aby nebyly moc těžké na pochopení. Programy například nebudou optimalizovány, aby měly co nejméně řádků nebo aby byly nejrychlejší. Komentáře jsou v kódu označeny \textcolor{commentgreen}{\#} a \textcolor{commentgreen}{tmavě zelenou barvou}.

% \section{Dijkstrův algortimus}
\label{sec:dijkstra_program}

Jelikož Dijkstrův algoritmus je součástí našeho algoritmu na řešení problému ve 2D, tak ho budu muset naprogramovat. Knihovna \href{https://networkx.org/}{networkx} má také k dispozici Dijkstrův algoritmus, ale pro náš problém je potřeba lehce modifikovaný; musí se zastavit v ten moment, kdy se vybere cílový bod za aktuální bod. 

\begin{lstlisting}[style=me]
def dijkstra_triangle(G:Graph, start, end):
    Q = set()
    distances = {} 
    middle_point = {} 
    for vertex in G:
        distances[vertex] = float('inf')
        middle_point[vertex] = None
        Q.add(vertex)

    distances[start] = 0
    while Q: 
        actual = min(Q, key=lambda vertex: distances[vertex])
        if actual == end:
            return distances[end], [start, middle_point[end], end]
        Q.remove(actual)
        neighbors = Q
        if actual == start:
            neighbors.remove(end)
        for neighbor in neighbors:
            calculated_distance = distances[actual] + G.get_edge_data(actual, neighbor)['weight']
            if calculated_distance < distances[neighbor]:
                distances[neighbor] = calculated_distance
                middle_point[neighbor] = actual
    return distances[end], [start, middle_point[end], end]
\end{lstlisting}
\section{Program v 1D}
\label{sec:program_1D}
Naprogramovat \myref{algoritmus}{alg:algoritmus_1D}, který řeší problém v dimenzi 1, je velmi jednoduchý. Využijeme funkci Pythonu sort(), která dokáže seřadit prvky v seznamu. Vstupem bude množina bodů a výstupem budou 2 body, které tvoří nejkratší úsečku

\begin{mdframed}[style=MyFrame]
\begin{lstlisting}[style=metoo]
def algorithm(V):
  points = list(V) # Seznam čísel.
  points.sort() # Seřazení seznamu čísel.
  shortest = float('inf') # Proměnná pro úsečku s minimální délkou.
  points = None # Proměnná pro body.
  for i in range(len(points-1)): # Cyklus procházení bodů.
    if abs(points[i] - points[i+1]) < shortest: # Pokud je úsečka menší.
      shortest = points[i] - points[i+1] # Aktualizujeme délku nejkratší úsečky.
      points = points[i], points[i+1] # Aktualizujeme nejkratší úsečku.
  return points
\end{lstlisting}
\end{mdframed}

\begin{center}
  

  \begin{tikzpicture}
    \centering
    % Define the dimensions of the table
    \def\cols{5}
    \def\cellsize{1}
    \def\colooor{myblue!60}
    \def\pivotcolor{mygreen!70}
    % Draw the table
    \draw [decorate,decoration={brace,amplitude=7.5pt}, color=\pivotcolor, line width=1pt](-1,-3) -- (0,-3) node[midway,yshift=15pt]{$1$};

    \foreach \clocklor [count=\col] in {\pivotcolor,\colooor, \colooor, \pivotcolor,\colooor} {
        \draw[fill=\clocklor, line width=1.5pt] (\col*\cellsize -3.5, -\cellsize+1) rectangle ++(\cellsize, -\cellsize);
            }
    % Label the cells
      \foreach \text [count=\col] in {-1, -4, 3, 0, 5}{
        \node at (\col*\cellsize + 0.5*\cellsize -3.5, 1 -\cellsize - 0.5*\cellsize) {\text};
      }

      \draw[-, line width=1.5pt] (-5.5,-3) -- (5.5,-3); % Number line
      \foreach \x in {-5,-4,-3,-2,-1,0,1,2,3,4,5} % Numbers
      \draw[line width=1.5pt] (\x,-2.9) -- (\x,-3.1) node[below] {\x};

      % \draw[-{Stealth[length=2mm]}, bend left=30] (0.5,-1) to (0,-3);
      % \draw[-{Stealth[length=2mm]}, bend left=30] (1.5,-1) to (5,-3);
      % \draw[-{Stealth[length=2mm]}, bend left=30] (2.5,-1) to (1,-3);
      % \draw[-{Stealth[length=2mm]}, bend left=30] (3.5,-1) to (-2,-3);
      % \draw[-{Stealth[length=2mm]}, bend left=30] (4.5,-1) to (0,-3);
      \draw[-{Stealth[length=2mm]}, bend left=30, line width=1pt] (0,-1) to (3,-2.9);
      \draw[-{Stealth[length=2mm]}, bend left=30, line width=1pt] (1,-1) to (0,-2.9);
      \draw[-{Stealth[length=2mm]}, bend left=30, line width=1pt] (2,-1) to (5,-2.9);
      \draw[-{Stealth[length=2mm]}, bend left=30, line width=1pt] (-1,-1) to (-4,-2.9);
      \draw[-{Stealth[length=2mm]}, bend left=30, line width=1pt] (-2,-1) to (-1,-2.9);
      % \draw[-{Stealth[length=2mm]}, bend left=30] (0.5,-1) to (0,-3);


  \end{tikzpicture}

  % \begin{tikzpicture}

  % \end{tikzpicture}
\end{center}
\section{Program ve 2D}
\label{sec:program_2D}

Ve 2D je program trochu těžší, protože hledáme 3 body, které tvoří trojúhelník s minimálním obvodem a musíme kontrolovat pouze jestli nejsou na přímce. Tento program dostane jako vstup množinu bodů, kde body budou v $\R^2$ a výstupem budou 3 body, které tvoří trojúhelník s minimálním obvodem. 
\begin{mdframed}[style=MyFrame]
\begin{lstlisting}[style=metoo]
from math import dist  # Importuje funkci dist z knihovny math pro výpočty.
from networkx import Graph, neighbors  # Importuje třídu Graph a funkci neighbors z knihovny networkx pro práci s grafy.

def find_path(G:Graph, start, end):
  min_path = None, None, None, float("inf")  # Proměnná pro nejkratší cestu s nekonečnou vzdáleností.
  for x in neighbors(G, start) :  # Prochází sousedy počátečního bodu v grafu G.
    path = G.get_edge_data(start, x)["weight"] + G.get_edge_data(end, x)["weight"]  #~Spočítá~vzdálenost~cesty~přes~aktuálního~souseda.
    if float(min_path[3]) > path:  # Pokud je nová cesta kratší než dosavadní.
      min_path = start, x, end, path  # Aktualizuje nejkratší cestu.
  return min_path  # Vrátí nejkratší cestu.

def algorithm(V:set):
  G = Graph()  # Prázdný graf.
  edges = []  # Prázdný seznam hran.
  while V:  # Dokud je množina V neprázdná.
    x = V.pop()  # Odebere prvek z množiny V.
    for v in V:  # Prochází zbývající prvky množiny V.
      if x!=v:  # Pokud se prvky liší.
        edges.append((x, v, dist(x, v)))# Přidá hranu s její váhou do seznamu hran.
  G.add_weighted_edges_from(edges)  # Přidá hrany do grafu G s příslušnými váhami.
  min_triangle = None  # Proměnná pro trojúhelník s minimální váhou.
  min_triangle_weight = float("inf")  # Proměnná pro minimální váhu trojúhelníku.
  for edge in edges:  # Prochází hrany v seznamu hran.
    edge_weight = G.get_edge_data(edge[0], edge[1])["weight"] # Váha hrany edge.
    G.remove_edge(edge[0], edge[1])  # Odebere hranu z grafu.
    u, j, v, weight= find_path(G, edge[0], edge[1])  # Najde nejkratší cestu mezi body edge[0] a edge[1].
    if (v[1]-u[1])*(j[0]-u[0]) == (j[1]-u[1])*(v[0]-u[0]):  # Pokud body u, j, v leží na jedné přímce.
      if dist(u, j) > dist(u, v) or dist(j, v) > dist(u, v):  # Pokud je vzdálenost mezi body u a j větší než vzdálenost mezi body u a v.
        G.remove_edge(u, j)  # Odebere hranu mezi body u a j.
      elif dist(j, v) > dist(u, v):  # Pokud je vzdálenost mezi body j a v větší než vzdálenost mezi body u a v.
        G.remove_edge(v, j)  # Odebere hranu mezi body v a j.
      else:
        continue  # Pokračuje na další iteraci cyklu.
    elif weight + edge_weight < min_triangle_weight:  # Pokud je váha cesty plus váha hrany menší než váha nejmenšího trojúhelníku.
      min_triangle = (u, j, v)  # Aktualizuje nejmenší trojúhelník.
      min_triangle_weight = weight + edge_weight  # Aktualizuje váhu nejmenšího trojúhelníku.
    G.add_weighted_edges_from([(u, v, dist(u, v))])  # Přidá hranu u, v zpátky do grafu s její váhou.

  return min_triangle, min_triangle_weight  # Vrátí trojúhelník a jeho obvod.
\end{lstlisting}
\end{mdframed}
\input{prakticka_cast/programovani/program_nD}