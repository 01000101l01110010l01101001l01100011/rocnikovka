\chapter{Programování algoritmů}
\label{chap:programovani}

Moje praktická část bude programování problému, o kterým jsem psal v teoretické části. K programování použiji programovací jazyk \href{https://www.python.org/}{python}, ve kterém už umím programovat. Python nabízí mnoho různých knihoven, které dokážou velmi ušetřit práci. Knihovna je sbírka funkcí a tříd, které jsou předem napsané a můžeme je získat jako doplněk pro snadnější vývoj softwaru. Ve své práci použiji knihovnu \href{https://networkx.org/}{networkx}, která je optimalizovaná pro práci s grafy. Použiji ji, protože nabízí lehké ukládání a získávání dat z grafu.

\section{Dijkstrův algortimus}
\label{sec:dijkstra_program}

Jelikož Dijkstrův algoritmus je součástí našeho algoritmu na řešení problému ve 2D, tak ho budu muset naprogramovat. Knihovna \href{https://networkx.org/}{networkx} má také k dispozici Dijkstrův algoritmus, ale pro náš problém je potřeba lehce modifikovaný; musí se zastavit v ten moment, kdy se vybere cílový bod za aktuální bod. 

\begin{lstlisting}[style=me]
def dijkstra_triangle(G:Graph, start, end):
    Q = set()
    distances = {} 
    middle_point = {} 
    for vertex in G:
        distances[vertex] = float('inf')
        middle_point[vertex] = None
        Q.add(vertex)

    distances[start] = 0
    while Q: 
        actual = min(Q, key=lambda vertex: distances[vertex])
        if actual == end:
            return distances[end], [start, middle_point[end], end]
        Q.remove(actual)
        neighbors = Q
        if actual == start:
            neighbors.remove(end)
        for neighbor in neighbors:
            calculated_distance = distances[actual] + G.get_edge_data(actual, neighbor)['weight']
            if calculated_distance < distances[neighbor]:
                distances[neighbor] = calculated_distance
                middle_point[neighbor] = actual
    return distances[end], [start, middle_point[end], end]
\end{lstlisting}
\section{Program ve 2D}
\label{sec:program_2D}

Ve 2D je program jednodušší, než ve více dimenzích, protože hledáme pouze 3 body, které jsou u sebe nejblíž a musíme kontrolovat pouze jestli nejsou na přímce. Ve vyšších dimenzích budeme muset kontrolovat, jestli body nejsou ve stejných rovinách nebo i nadrovinách. Tento program dostane vstup množinu bodů, kde body budou v $\R^2$ a výstupem budou 3 body, které tvoří trojúhelník s minimálním obvodem. 

V tomto programu použiji Dijkstrův algoritmus, který jsem naprogramoval v \myref{sekci}{sec:dijkstra_program}. 

\textcolor{red}{tohle jde udelat jeste jinak at tam neni to edges v tom algoritmu}
\begin{lstlisting}[style=metoo]
import dijkstra

def find_shortest_path(G:Graph, edges):
    shortest_path = None
    smallest_triangle = float('inf')
    for edge in G.edges:
        G.remove_edge(*edge)
        path_length, path = dijkstra.dijkstra_triangle(G, *edge)
        triangle = edges[edge] + path_length
        if triangle  < smallest_triangle:
            shortest_path = path
            smallest_triangle = triangle
        G.add_edge(*edge, weight=edges[edge])
    return shortest_path
\end{lstlisting}