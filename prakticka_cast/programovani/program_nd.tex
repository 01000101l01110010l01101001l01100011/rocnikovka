\section{Program v \texorpdfstring{$n$D}{nD}}
\label{sec:program_nD}

V $n$D program vytvoří všechny $(n+1)$-prvkové podmnožiny z množiny bodů $V$. Poté program spočítá obvod polytopu. Je-li menší, než ten uložený, ověří, že má maximální dimenzi. Pokud ano, uloží ho jako minimální.

\begin{mdframed}[style=MyFrame]
\begin{lstlisting}[style=metoo]
import numpy as np
from itertools import combinations
from scipy.spatial.distance import euclidean # Import knihoven.

def generate_subsets(points, subset_length):
  # Všechny (n+1)-prvkové podmnožiny.
  subsets = combinations(points, subset_length)
  return subsets


def algorithmnd(points):
  num_points = len(points)  # Počet bodů.
  point_indices = list(range(num_points))  # Indexy bodů.
  n = len(points[0])   # Dimenze.
  # Vytvoří všechny (n+1)-prvkové podmnožiny.
  subsets = list(generate_subsets(point_indices, n+1))

  # Vytvoří matici vzdáleností.
  distances = np.zeros((num_points, num_points))
  for i in range(num_points):
    for j in range(i+1, num_points):
      distances[i, j] = euclidean(points[i], points[j])
      distances[j, i] = distances[i, j]

  polytopes = []  # Seznam polytopů.
  for subset in subsets:  # Pro podmnožinu (polytop).
    perimeter = 0
    edges = generate_subsets(subset, 2)
    for edge in edges:
      perimeter += distances[edge[0], edge[1]]  # Z matice vzdáleností spočítá obvod polytopu.
    current_points = [points[j] for j in subset]
    polytopes.append((current_points, perimeter))
  sorted_polytopes = sorted(polytopes, key=lambda x: x[-1])
  while len(sorted_polytopes) > 0:
    polytope, distance = sorted_polytopes[0]  # Polytop a jeho obvod.
    x1 = polytope[0]  # První bod polytopu.
    differences = []  # Seznam rozdílů.
    for x in polytope[1:]:  # Pro bod v polytopu.
      difference = x - x1  # Rozdíl bodů.
      differences.append(difference)  # Přidá rozdíl do seznamu rozdílů.
    matrix = np.vstack(differences)  # Vytvoří matici z rozdílů.
    determinant = np.linalg.det(matrix)  # Spočítá determinant.
    if determinant != 0:  # Pokud je determinant nenulový, řešením je tento polytop.
      return polytope, distance
\end{lstlisting}
\end{mdframed}