\section{Program v \texorpdfstring{$n$D}{nD}}
\label{sec:program_nD}

Ve $2$D program vytvoří všechny $(n+1)$-prvkové podmnožiny z množiny bodů $V$. Poté program spočítá obvod polytopu a pokud je menší, než ten uložený, ověří, že se opravdu jedná o polytop, a pokud ano, uloží tento polytop jako minimální polytop.

\begin{mdframed}[style=MyFrame]
\begin{lstlisting}[style=metoo]
num_points = len(points)  # Počet bodů.
n = len(points[0])   # Dimenze.
subsets = generate_subsets(points, n+1) # Vytvoří všechny (n+1)-prvkové podmnožiny.
polytopes = [] # Seznam polytopů.
for subset in subsets  # Pro podmnožinu (polytop).
  perimeter = 0
  edges = generate_subsets(subset, 2) # Množina všech 2-prvkových podmnožin.
  for edge in edges:
    perimeter += euclidean(edge[0], edge[1]) # Délka hrany.
  polytopes.append((subset, perimeter)) # Přidá hranu do seznamu polytopů.
sorted_polytopes = sorted(polytopes, key=lambda x: x[-1]) # Seřadí seznam polytopů podle jejich obvodu.
while len(sorted_polytopes) > 0:
  polytope, distance = sorted_polytopes[0] # Polytop a jeho obvod.
  x1 = polytope[0] # První bod polytopu.
  differences = [] # Seznam rozdílů.
  for x in polytope[1:]: # Pro bod v polytopu.
    difference = x - x1 # Rozdíl bodů.
    differences.append(difference) # Přidá rozdíl do seznamu rozdílů.
  matrix = np.vstack(differences) # Vytvoří matici z rozdílů.
  determinant = np.linalg.det(matrix) # Spočítá determinant.
  if determinant != 0: # Pokud je determinant nenulový, řešením je tento polytop.
    return polytope, distance
\end{lstlisting}
\end{mdframed}