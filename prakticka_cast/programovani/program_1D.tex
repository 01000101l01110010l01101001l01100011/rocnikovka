\section{Program v 1D}
\label{sec:program_1D}
Naprogramovat \myref{algoritmus}{alg:algoritmus_1D}, který řeší problém v dimenzi 1, je velmi jednoduchý. Využijeme funkci sort() z knihovny NumPy, která dokáže seřadit prvky v seznamu. Vstupem bude množina bodů a výstupem budou dva body, které tvoří nejkratší úsečku.

\begin{mdframed}[style=MyFrame]
\begin{lstlisting}[style=metoo]
def algorithm(points:np.array):
points = np.sort(points, kind='quicksort') # Seřazení seznamu čísel
  shortest = float('inf') # Proměnná pro úsečku s minimální délkou.
  for i in range(len(points)-1): # Cyklus procházení bodů.
    if abs(points[i] - points[i+1]) < shortest: # Pokud je úsečka menší.
      shortest = points[i] - points[i+1] # Aktualizujeme délku nejkratší úsečky.
      closest_points = points[i], points[i+1] # Aktualizujeme nejkratší úsečku.
  return closest_points
\end{lstlisting}
\end{mdframed}
\begin{center}
\begin{figure}[H]
  \centering
  \begin{tikzpicture}
    \centering
    % Define the dimensions of the table
    \def\cols{5}
    \def\cellsize{1}
    \def\colooor{myblue!60}
    \def\pivotcolor{mygreen!70}
    % Draw the table
    \draw [decorate,decoration={brace,amplitude=7.5pt}, color=\pivotcolor, line width=1pt](-1,-3) -- (0,-3) node[midway,yshift=15pt]{$1$};

    \foreach \clocklor [count=\col] in {\pivotcolor,\colooor, \colooor, \pivotcolor,\colooor} {
        \draw[fill=\clocklor, line width=1.5pt] (\col*\cellsize -3.5, -\cellsize+1) rectangle ++(\cellsize, -\cellsize);
            }
    % Label the cells
      \foreach \text [count=\col] in {-1, -4, 3, 0, 5}{
        \node at (\col*\cellsize + 0.5*\cellsize -3.5, 1 -\cellsize - 0.5*\cellsize) {\text};
      }

      \draw[-, line width=1.5pt] (-5.5,-3) -- (5.5,-3); % Number line
      \foreach \x in {-5,-4,-3,-2,-1,0,1,2,3,4,5} % Numbers
      \draw[line width=1.5pt] (\x,-2.9) -- (\x,-3.1) node[below] {\x};

      % \draw[-{Stealth[length=2mm]}, bend left=30] (0.5,-1) to (0,-3);
      % \draw[-{Stealth[length=2mm]}, bend left=30] (1.5,-1) to (5,-3);
      % \draw[-{Stealth[length=2mm]}, bend left=30] (2.5,-1) to (1,-3);
      % \draw[-{Stealth[length=2mm]}, bend left=30] (3.5,-1) to (-2,-3);
      % \draw[-{Stealth[length=2mm]}, bend left=30] (4.5,-1) to (0,-3);
      \draw[-{Stealth[length=2mm]}, bend left=30, line width=1pt] (0,-1) to (3,-2.9);
      \draw[-{Stealth[length=2mm]}, bend left=30, line width=1pt] (1,-1) to (0,-2.9);
      \draw[-{Stealth[length=2mm]}, bend left=30, line width=1pt] (2,-1) to (5,-2.9);
      \draw[-{Stealth[length=2mm]}, bend left=30, line width=1pt] (-1,-1) to (-4,-2.9);
      \draw[-{Stealth[length=2mm]}, bend left=30, line width=1pt] (-2,-1) to (-1,-2.9);
      % \draw[-{Stealth[length=2mm]}, bend left=30] (0.5,-1) to (0,-3);


  \end{tikzpicture}
  \caption{Příklad, kdy algoritmus seřadil čísla a vybral nejkratší úsečku.} \label{obrazek:algoritmus_2D}
  \end{figure}
\end{center}