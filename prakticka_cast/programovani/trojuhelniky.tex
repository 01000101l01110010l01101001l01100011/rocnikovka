\section{Program ve 2D}
\label{sec:program_2D}

Ve 2D je program jednodušší, než ve více dimenzích, protože hledáme pouze 3 body, které jsou u sebe nejblíž a musíme kontrolovat pouze jestli nejsou na přímce. Ve vyšších dimenzích budeme muset kontrolovat, jestli body nejsou ve stejných rovinách nebo i nadrovinách. Tento program dostane vstup množinu bodů, kde body budou v $\R^2$ a výstupem budou 3 body, které tvoří trojúhelník s minimálním obvodem. 

V tomto programu použiji Dijkstrův algoritmus, který jsem naprogramoval v \myref{sekci}{sec:dijkstra_program}. 

\textcolor{red}{tohle jde udelat jeste jinak at tam neni to edges v tom algoritmu}
\begin{lstlisting}[style=metoo]
import dijkstra

def find_shortest_path(G:Graph, edges):
    shortest_path = None
    smallest_triangle = float('inf')
    for edge in G.edges:
        G.remove_edge(*edge)
        path_length, path = dijkstra.dijkstra_triangle(G, *edge)
        triangle = edges[edge] + path_length
        if triangle  < smallest_triangle:
            shortest_path = path
            smallest_triangle = triangle
        G.add_edge(*edge, weight=edges[edge])
    return shortest_path
\end{lstlisting}