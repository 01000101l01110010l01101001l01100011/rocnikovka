\chapter*{Závěr}
% Závěr není číslovaný, třeba ho do obsahu přidat manuálně.
\addcontentsline{toc}{chapter}{Závěr}

V této práci jsme hledali polytopy maximálních dimenzí s minimálním obvodem v různých dimenzích. Jako první čtenář mohl vidět základní definice a použitou notaci. Na definice se obracím v textu a předpokládám, že čtenář je s nimi seznámen. 

Problém jsme si rozdělili na 1D, kde jsme hledali nejkratší úsečku, pak jsme ve 2D hledali trojúhelník s minimálním obvodem  a nakonec jsme problém zobecnili na $n$ dimenzí. Problém ve 2D by nejspíš šel urychlit tím, že bychom zakazovali cesty, které jsou větší, než obvod nejmenšího trojúhelníku. Všechny algoritmy jsme dokázali, na druhou stranu, nebylo to těžké je dokázat, protože to bylo zkoušení všech možností. Aspoň algoritmus v $n$D je v průměru rychlejší, protože ověřujeme, jestli polytop tvoří maximální dimenzi až na konec od polytopu s nejmenším obvodem. V nejlepším případě se determinant matice, který je časově náročný, spočítá pouze jednou. V nejhorším případě se může stát, že pouze polytop, který má nejdelší obvod, má maximální dimenzi. 

V praktické části jsem naprogramoval algoritmy v Pythonu. Kód je okomentovaný, ale už jsem ho znovu nevysvětloval, protože  princip algoritmu jsem vysvětlil v teoretické části. 
